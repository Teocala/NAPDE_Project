\documentclass[9pt]{beamer}
\mode<presentation>
{
	\usetheme{Warsaw}
	\usecolortheme{beaver}
}

\usepackage[english]{babel}
\usepackage[utf8]{inputenc}
\usepackage[T1]{fontenc}
\usepackage{times}
\usepackage{media9}
\usepackage{color}
\usepackage{physics}
\usepackage{amsmath}
\usepackage{amsfonts}


\newtheorem*{remark}{Remark}
\newtheorem*{theor}{Theorem}

\title[\color{white}{Project N°2}]{A HIGH-ORDER DISCONTINUOUS GALERKIN METHOD FOR THE BIDOMAIN PROBLEM OF CARDIAC ELECTROPHYSIOLOGY}
\subtitle{Project N$^\circ$ 2}
\author[]{\small{\textit{Supervisors: Christian Vergara, Paola Antonietti}}\\ \vspace{4mm} Federica Botta, Matteo Calafà}
\institute[Politecnico di Milano]{\scriptsize{Course of Numerical Analysis for Partial Differential Equations}}
\date{\tiny{A.Y. 2020/2021}}


\logo{\includegraphics[height=0.5cm]{logo_polimi.png}}



\begin{document}

\frame{\titlepage}

\begin{frame}
\frametitle{Table of Contents}
\tableofcontents
\end{frame}

\begin{frame}
\section{Introduction}
\frametitle{The physical problem}
\begin{center}
Mechanical contraction of the human heart\\

$\uparrow$

Electrical activation of the cardiac cells\\

$\downarrow$

Continuous electrical diffusion over the entire cardiac surface.\\
\end{center}
\begin{columns}
            \begin{column}{0.5\textwidth}
                  \begin{figure}[t]
                  \includegraphics[width = 0.8\textwidth]{./potential_cycle.png}
                  \centering
                  \end{figure}
            \end{column}
            \begin{column}{0.5\textwidth}  
            %\begin{figure}[t]
            %      \includegraphics[width = 0.8\textwidth]{./image.png}
            %      \centering
            %      \end{figure} 
                  \includemedia[width=5cm,height=3cm,addresource=Video_Presentazione.mp4,activate=pageopen, flashvars={
                  	source=Video_Presentazione.mp4 &loop=true
                  }]{}{VPlayer.swf}
                  \centering
            \end{column}
     \end{columns}
\end{frame}


\begin{frame}
\frametitle{The mathematical model}
\center
\underline{Bidomain model + FitzHugh-Nagumo with Neumann B.C.}
\begin{equation*}
\small
\begin{cases}
\chi_m C_m\pdv{V_m}{t} - \nabla \cdot (\Sigma_i \nabla \phi_i) + \chi_m I_{ion}(V_m,w) = I_i^{ext},    & \text{in } \Omega_{mus} \cross (0,T],
\\
-\chi_m C_m\pdv{V_m}{t} - \nabla \cdot (\Sigma_e \nabla \phi_e) - \chi_m I_{ion}(V_m,w) = -I_e^{ext},    & \text{in } \Omega_{mus} \cross (0,T],
\\
I_{ion}(V_m,w)=kV_m(V_m-a)(V_m-1)+w, & \text{in } \Omega_{mus} \cross (0,T],
\\
\pdv{w}{t} = \epsilon(V_m-\gamma w),  & \text{in } \Omega_{mus} \cross (0,T],
\\
\Sigma_i\nabla \phi_i \cdot n = b_i,   & \text{on } \partial \Omega_{mus} \cross (0,T],
\\
\Sigma_e\nabla \phi_e \cdot n = b_e,   & \text{on } \partial \Omega_{mus} \cross (0,T],
\\
\text{Initial conditions for } \phi_i,\phi_e, w, & \text{in } \Omega_{mus}\cross\{t=0\}.
\end{cases}
\end{equation*}
\center \small
Unknowns: $\phi_i,\,\phi_e,\, V_m=\phi_i-\phi_e,\, w$
\end{frame}


\begin{frame}
\frametitle{Our objectives}

What had already been done:
\begin{itemize}
	\item Implementation of a Discontinuous Galerkin with FEM basis for the Bidomain problem.
	\item Implementation of a Semi-Implicit temporal scheme.
\end{itemize} \vspace{2mm}
What we did:
\begin{itemize}
	\item Implementation of a Discontinuous Galerkin with \textbf{Dubiner} basis for the Bidomain problem.
	\item Implementation of further temporal schemes.
	\item Bugs corrections and optimizations.
	\item Pseudo-realistic simulations.
\end{itemize}
\end{frame}


\begin{frame}
\section{Dubiner basis implementation}
\frametitle{Analytical definition}
\begin{definition}[Dubiner basis] \label{dubiner}
	The Dubiner basis that generates the space $\mathbb{P}^p(\hat{K})$ of polynomials of degree $p$ over the reference triangle is the set of functions:
	\begin{equation*}
	\begin{gathered}
	 \phi_{ij}: \hat{K} \rightarrow \mathbb{R}, \\
	\phi_{ij}(\xi,\eta) =c_{ij} \, 2^j (1-\eta)^j J_i^{0,0}(\frac{2\xi}{1-\eta}-1) J_j^{2i+1,0} (2\eta-1),
	\end{gathered}
	\end{equation*}
	\vspace{2mm} \\
	\center
	for $i,j=0,\dots,p$ and $i+j \le p$, where $c_{ij} := \sqrt{\frac{2(2i+1)(i+j+1)}{4^i}}$ \\
	and $J_i^{\alpha,\beta}(\cdot)$ is the i-th Jacobi polynomial.
\end{definition}
\end{frame}

\begin{frame}
\frametitle{Properties}
\begin{itemize}
	\item They consist in a pseudo tensor-product of Jacobi polynomials if the following transformation is then applied:
	\begin{columns}
		\begin{column}{0.5\textwidth}
			\begin{equation*}\label{transformation_formula}
			\quad \quad \quad \xi=\frac{(1+a)(1-b)}{4},  \eta=\frac{(1+b)}{2}.
			\end{equation*}
		\end{column}
		\begin{column}{0.5\textwidth}  
			\includegraphics[width = 4cm]{./transformation.png}
		\end{column}
	\end{columns}
    \item They are $L^2(\hat{K})$ orthonormal ($\hat{K}$ is the reference triangle).
\end{itemize}
\end{frame}


\begin{frame}
\frametitle{Main works}
\begin{remark}
	Dubiner basis coefficients of a discretized function have \textbf{modal} meaning instead of a \emph{nodal} meaning.
\end{remark}
Then, our main works regarded:
\begin{itemize}
	\item Methods for the evaluation of the Dubiner functions and gradients in the reference points.
	\item Methods for the evaluation of the FEM coefficients of a discretized function starting from its Dubiner coefficients and viceversa.
	\begin{enumerate}
		\item FEM $\rightarrow$ Dubiner is needed when we use the initial solution data into the Dubiner system.
		\item FEM $\leftarrow$ Dubiner is needed when we want to get the solution obtained from the Dubiner system in a comprehensible form. 
	\end{enumerate}
\end{itemize}
\end{frame}


\begin{frame}
\frametitle{FEM-Dubiner conversion strategies}
Consider:
\begin{itemize}
	\small
	\item An element $\mathcal{K}\in \tau_h$
	\item $\{\psi_{i}\}_{i=1}^{p}$,$\{\varphi_{j}\}_{j=1}^{q}$ as the FEM and Dubiner functions with support in $\mathcal{K}$. 
	\item $\{\hat{u}_i\}_{i=1}^p$,$\{\tilde{u}_j\}_{j=1}^q$ as the FEM and Dubiner coefficients of a function $u_h$.
\end{itemize} \vspace{4mm}
\textbf{FEM $\leftarrow$ Dubiner} \\
Exploiting the nodal meaning of FEM, we compute its value in a point:
\begin{equation*} \label{ref3}
\hat{u}_i = \sum_{j=1}^q \tilde{u}_j\phi_j(x_i),
\end{equation*}
\textbf{FEM $\rightarrow$ Dubiner} \\
Exploiting the $L^2$-orthonormality of Dubiner, we compute its Fourier coeff.:
\begin{equation*}\label{ref4}
\tilde{u}_j = \int_\mathcal{K} u_h(x) \varphi_j(x) \,dx = \int_{\mathcal{K}} \sum_{i=1}^p \hat{u}_i\psi_i(x) \varphi_j(x) \,dx = \sum_{i=1}^p \Big(\int_{\mathcal{K}}\psi_i(x)\varphi_j(x)\,dx \Big) \hat{u}_i.
\end{equation*}
\end{frame}

\begin{frame}
\section{Semi-discrete Discontinuous Galerkin}
\frametitle{Discretization}
$\qquad \qquad \qquad \qquad $ \textbf{space-dependent:} Discontinuous Galerkin method\\\vspace{2mm}
$\qquad \qquad \qquad \qquad \nearrow$\\
Bidomain problem\\
$\qquad \qquad \qquad \qquad \searrow$\\\vspace{2mm}
$\qquad \qquad \qquad \qquad $ \textbf{time-dependent:} Semi-implicit, Godunov\\ $\qquad \qquad \qquad \qquad $ operator-splitting and quasi-implicit operator-splitting
\end{frame}

\begin{frame}
\frametitle{Semi-discrete Discontinuous Galerkin formulation}
For any $t\in[0,T]$ find $\Phi_h(t)=[\phi_i^h(t),\phi_e^h(t)]^T \in [V_h^p]^2$  and  $w_h(t) \in V_h^p$ such that:
\vspace{2mm}
    \begin{equation*}
    \begin{cases}
    \begin{gathered}
    \sum_{\mathcal{K} \in \tau_h} \int_{\mathcal{K}}{ \chi_m C_m \pdv{V_m^h}{t} v_h d\omega}+a_i(\phi_i^h,v_h)+\sum_{\mathcal{K} \in \tau_h} \int_{\mathcal{K}}{ \chi_m k (V_m^h-1)(V_m^h-a) V_m^h v_h d\omega}+\\
    +\sum_{\mathcal{K} \in \tau_h} \int_{\mathcal{K}}{ \chi_m w_h v_h d\omega}=(I_i^{ext},v_h), \qquad \forall v_h \in V_h^p,\\
    \end{gathered}\\
    \vspace{2mm}
    \begin{gathered}
    -\sum_{\mathcal{K} \in \tau_h} \int_{\mathcal{K}}{ \chi_m C_m \pdv{V_m^h}{t} v_h d\omega}+a_e(\phi_e^h,v_h)-\sum_{\mathcal{K} \in \tau_h} \int_{\mathcal{K}}{ \chi_m k (V_m^h-1)(V_m^h-a) V_m^h v_h d\omega}+\\
    -\sum_{\mathcal{K} \in \tau_h} \int_{\mathcal{K}}{ \chi_m w_h v_h d\omega}=(-I_e^{ext},v_h), \qquad \forall v_h \in V_h^p,\\
    \end{gathered}\\
    \vspace{2mm}
    \sum_{\mathcal{K} \in \tau_h} \int_{\mathcal{K}}{\pdv{w_h}{t}v_h d\omega}=\sum_{\mathcal{K} \in \tau_h} \int_{\mathcal{K}}{\epsilon (V_m^h-\gamma w_h) v_h d\omega}, \qquad \forall v_h \in V_h^p,\\
    \end{cases}
    \end{equation*}
\end{frame}
\begin{frame}
    where:
\center
    \begin{equation*}
    \begin{aligned}
    \bullet& \quad a_l(\phi_l^h,v_h)=\sum_{\mathcal{K} \in \tau_h} \int_{\mathcal{K}}{(\Sigma_l \nabla_h \phi_k^h) \cdot \nabla_h v_h d\omega}-\sum_{F \in \mathcal{F}_h^I} \int_F { \{\{\Sigma_l \nabla_h \phi_l^h \}\} \cdot [[v_h]] d\sigma}+\\
    &-\delta \sum_{F \in \mathcal{F}_h^I} \int_F{ \{\{\Sigma_l \nabla_h v_h\}\} \cdot [[\phi_l^h]]d\sigma}+\sum_{F \in \mathcal{F}_h^I}\int_F {\Gamma [[\phi_l^h]] \cdot [[v_h]] d\sigma} \qquad l=i,e,\\
    \newline
    \bullet& \quad (I_i^{ext},v_h)=\sum_{\mathcal{K} \in \tau_h} \int_{\mathcal{K}} {I_i^{ext} v_h d\omega}+\int_{\partial \Omega}{b_i v_h d\sigma},\\
    \newline
    \bullet& \quad (-I_e^{ext},v_h)=-\sum_{\mathcal{K} \in \tau_h} \int_{\mathcal{K}} {I_e^{ext} v_h d\omega}+\int_{\partial \Omega}{b_e v_h d\sigma}.
    \end{aligned}
    \end{equation*}
\end{frame}

\begin{frame}
\section{Temporal discretization}
\frametitle{Semi-implicit scheme}
Idea:
\begin{itemize}
\item treat most of the terms of the PDE implicitly,
\item treat the non-linear term semi-implictly,
\item treat the ODE implictly with the exception of the term $V_m$.
\end{itemize}
\vspace{3mm}
\begin{center}
\underline{Semi-implicit discretized system}
\end{center}
Find $\Phi^{n+1}=[\phi_i^{n+1} \phi_e^{n+1}]^T$ and $w^{n+1}$ $\forall n=0,\cdots,N-1$ such that:
\begin{equation*}
\begin{cases}
(\frac{1}{\Delta t}+\epsilon \gamma)M w^{n+1}=\epsilon M V_m^n+\frac{M}{\Delta t} w^n, \vspace{2mm} \\
(B+C_{nl}(V_m^n)) \Phi^{n+1}=r^{n+1}.
\end{cases}
\end{equation*} 
\end{frame}

\begin{frame}
\frametitle{Godunov operator-splitting scheme}
The main feature is the sub-division of the problem into two different problems to be solved sequentially, such that $L(u)=L_1(u)+L_2(u)$. In our case:
\vspace{2mm}
\begin{columns}
            \begin{column}{0.5\textwidth}
\small 1:
 \begin{equation*}
\begin{cases}
\chi_m C_m M\frac{\hat{V}_m^{n+1} - V_m^n}{\Delta t} + C(V_m^n)V_m^n +\chi_mMw^n = 0,\vspace{2mm}\\
\frac{w^{n+1}-w^n}{\Delta t} = \epsilon (V_m^n - \gamma w^n).
\end{cases}
\end{equation*}
            \end{column}
            \hspace{1cm}
            \begin{column}{0.5\textwidth}  
            \small 2:
\begin{equation*}
\begin{cases}
\chi_m C_m M\frac{V_m^{n+1} - \hat{V}_m^{n+1}}{\Delta t} + A_i\phi_i^{n+1} = F_i^{n+1}, \vspace{2mm}\\
-\chi_m C_m M\frac{V_m^{n+1} - \hat{V}_m^{n+1}}{\Delta t} + A_e\phi_e^{n+1} = F_e^{n+1}.
\end{cases}
\end{equation*}

            \end{column}
     \end{columns}
     \vspace{4mm}
\begin{center}
\underline{Godunov operator-splitting discretized system}
\end{center}
Find $\Phi^{n+1}=[\phi_i^{n+1} \phi_e^{n+1}]^T$ and $w^{n+1} \quad \forall n=0, \cdots, N-1$ such that: \small
\begin{equation*}
\begin{cases}
\begin{gathered}
\left(
\frac{\chi_m C_m}{\Delta t} \begin{bmatrix}M & -M \\ M & -M\end{bmatrix}
+ \begin{bmatrix} A_i & 0 \\ 0 & -A_e \end{bmatrix}
\right) \begin{bmatrix} \phi_i^{n+1} \\ \phi_e^{n+1}  \end{bmatrix} =
\begin{bmatrix} F_i^{n+1} \\ -F_e^{n+1} \end{bmatrix} + \\ -
\chi_m\begin{bmatrix} M & 0 \\ 0 & M \end{bmatrix} \begin{bmatrix} w^n \\ w^n \end{bmatrix} +
\left(\frac{\chi_mC_m}{\Delta t}\begin{bmatrix} M & 0 \\ 0 & M \end{bmatrix}
- \begin{bmatrix} C(V_m^n) & 0 \\ 0 & C(V_m^n)\end{bmatrix} 
\right) \begin{bmatrix} V_m^n \\ V_m^n \end {bmatrix},
\end{gathered} \\ \\
w^{n+1} = (1-\epsilon \gamma \Delta t) w^n + \epsilon \Delta tV_m^n.
\end{cases}
\end{equation*}

\end{frame}

\begin{frame}
\frametitle{Quasi-implicit operator-splitting scheme}
Idea:
\begin{itemize}
\item sub-division of the operator as Godunov operator-splitting
\item treat implicitly all the terms except the cubic one
\end{itemize}
\vspace{2mm}
In this case:
\begin{columns}
            \begin{column}{0.5\textwidth}
\small 1:
 \begin{equation*}
\begin{cases}
\chi_m C_m M \frac{\tilde{V}_m^{n+1}-V_m^n}{\Delta t} +  C(V_m^n) V_m^{n+1} + \chi_m M w^{n+1}= 0,\vspace{2mm}\\
\frac{w^{n+1} - w^n}{\Delta t} = \epsilon (V_m^{n+1}-\gamma w^{n+1}).
\end{cases}
\end{equation*}
            \end{column}
            \hspace{1cm}
            \begin{column}{0.5\textwidth}  
            \small 2:
\begin{equation*}
\begin{cases}
\chi_m C_m M \frac{V_m^{n+1}-\tilde{V}_m^{n+1}}{\Delta t} + A_i \phi_i^{n+1}= F_i^{n+1},\vspace{2mm}\\
- \chi_m C_m M \frac{V_m^{n+1}-\tilde{V}_m^{n+1}}{\Delta t} + A_e \phi_e^{n+1}= F_e^{n+1}.
\end{cases}
\end{equation*}
            \end{column}
     \end{columns}
\vspace{3mm}
\begin{center}
\underline{Quasi-implicit operator-splitting discretized system}
\end{center}
Find $\Phi^{n+1}=[\phi_i^{n+1} \phi_e^{n+1}]^T$ and $w^{n+1} \quad \forall n=0, \cdots, N-1$ such that:
\begin{equation*}
\quad
\begin{cases}
\left(
\begin{bmatrix} Q_n & -Q_n \\ Q_n & -Q_n \end{bmatrix} + 
\begin{bmatrix} A_i & 0 \\ 0 & -A_e\end{bmatrix}
\right)
\begin{bmatrix}
\phi_i^{n+1} \\ \phi_e^{n+1}
\end{bmatrix}
= \begin{bmatrix} R_n \\ R_n \end{bmatrix} + \begin{bmatrix} F_i^{n+1} \\  -F_e^{n+1}\end{bmatrix},\\ \\
w^{n+1} = \frac{\displaystyle w^n + \epsilon \Delta t (\phi_i^{n+1}-\phi_e^{n+1})}{\displaystyle 1+\epsilon \gamma \Delta t}.
\end{cases}
\end{equation*}
\end{frame}

\begin{frame}
\section{Uniqueness of the potentials}
\frametitle{Uniqueness}
About uniqueness of the unkowns:
\vspace{2mm}
\begin{itemize}
\item $V_m, w$ proved in \textit{Existence and uniqueness of the solution for the bidomain model used in cardiac electrophysiology} by Y. Bourgault, Y. Coudière, and C. Pierre.
\vspace{2mm}
\item $\phi_i, \phi_e$ appear only through their difference $V_m$ or their gradient. This means that there cannot be uniqueness.
\end{itemize}
\end{frame}
\begin{frame}
\frametitle{Uniqueness of potentials}
\begin{theor}
The classical solutions $\phi_i,\phi_e$ are unique up to a constant depending only on time.
\end{theor}
Namely:\\
Suppose now there exist two couples $(\phi_i^1,\phi_e^1)$,$(\phi_i^2,\phi_e^2)$ of potentials solutions of the Bidomain problem.
\begin{equation*}
\exists \tilde{\varphi}:[0,T]\rightarrow \mathbb{R} \text{ such that } \phi_i^1(x,t)-\phi_i^2(x,t) = \phi_e^1(x,t)-\phi_e^2(x,t)= \tilde{\varphi}(t)
\end{equation*} 
$\qquad \forall x \in \Omega,\forall t \in [0,T].$
\begin{center}
\textbf{SO HOW TO RESOLVE IT?}
\end{center}
\begin{enumerate}
\item Imposition of the value of the function in a specific point.
\item Imposition of the function mean value.
\end{enumerate}
\end{frame}

\begin{frame}
\frametitle{Imposition of a value in a specific point}
Focus on only one of the potentials $\phi_i$.\\
\vspace{3mm}
\begin{small}
$\phi_i(\bar{x},t) = \varphi(t) \quad \forall t \in [0,T] \qquad
\xrightarrow[\text{with } \{u_j\} \text{ as vector of solution}]{\text{Numerical version}}\quad
u_l^n = \varphi(t^n) \quad \forall n \in \{1,N\}$
\end{small}
\vspace{2mm}
\end{frame}
\end{document}
