\documentclass[a4paper,11pt]{article}
\usepackage[a4paper,top=4cm,bottom=4cm,left=3.2cm,right=3.2cm]{geometry}
\usepackage[T1]{fontenc}
\usepackage[utf8]{inputenc}
\usepackage[english]{babel}
\usepackage{graphicx}
\usepackage{amsmath}
\usepackage{amsfonts}
\usepackage{amsthm}
\usepackage{enumerate}
\usepackage{enumitem}
\usepackage{physics}
\usepackage{bm}
\usepackage{setspace}
\usepackage{caption}
\usepackage{mathtools}
\usepackage{subcaption}
\usepackage{relsize}
\usepackage{listings}
\captionsetup{compatibility=false}

\usepackage[
backend=biber,
style=numeric,
citestyle=numeric,
]{biblatex}

\addbibresource{./report_bib.bib}
\usepackage[autostyle]{csquotes}

\captionsetup{font=footnotesize}
\renewcommand*{\lstlistlistingname}{Codes}


\newtheorem{definition}{Definition}
\newtheorem{theo}{Theorem}
\newtheorem{prop}{Proposition} 
\newtheorem{problem}{Problem}
\newtheorem*{remark}{Remark}
\newtheorem{corollary}{Corollary}
\newtheorem{lemma}{Lemma}
\def\proof{\paragraph{Proof. }}


\begin{document}
	
	%% FRONT PAGE
	\begin{titlepage}
	    \thispagestyle{empty}
	    \newgeometry{left=2cm,right=2cm}
	    \begin{center}
	    	\includegraphics[width = 4cm]{./polimi-logo.png}\\ \vspace{3mm}
	    	\normalsize{\textsc{Course of Numerical Analysis for Partial Differential Equations}}
	    	
	    	\vspace{20mm}
	    	\rule{15cm}{0.1mm} \\ \vspace{4.5mm}
	    	 \Huge{\textbf{A HIGH-ORDER DISCONTINUOUS GALERKIN METHOD FOR THE BIDOMAIN PROBLEM OF CARDIAC ELECTROPHYSIOLOGY}} \\
	    	\rule{15cm}{0.1mm}
	    \end{center}
	    	\vspace{25mm}
	    	
	    	\Large{
	    	\hspace{11mm} \emph{Authors:} \hspace{5mm} \textsc{ \quad Federica Botta, Matteo Calafà}} \newline
	    	 
	    	\Large{\vspace{3mm} 
	    	 \hspace{4mm} \emph{Supervisors:} \hspace{3mm} \textsc{ Christian Vergara, Paola Antonietti}}
	          \\
	    	\vspace{20mm}
	    \begin{center}
	    	\large{\textsc{A.Y. 2020/2021}}
	    \end{center}
	\end{titlepage}



    \restoregeometry
    
    %% TABLE OF CONTENTS
    \tableofcontents
    \texttt{\lstlistoflistings}
    \newpage
    
    %% ABSTRACT
    \section{Introduction}
    \subsection{Abstract}
    The aim of the project is to study and implement a suitable numerical scheme for the resolution of the \emph{Bidomain Problem}, a famous system of equations that has been developed in the context of the electrophysiology of human heart. \\
    This work is basically the continuation of a two-years-long study carried out by three past course projects (\cite{bagnara}, \cite{andreotti}, \cite{marta}). In particular, the very goal of this project is to improve the results obtained in \parencite{marta} (\citeauthor{marta}) for the Bidomain model. In fact, even if a \emph{Discontinuous Galerkin} discretization has been successfully implemented, results are not satisfactory from the point of view of stability and convergence. We think this notice is noteworthy as this work is primarily based on these provided data and \textsc{Matlab}$^ \copyright$ codes. Through this article, it will be illustrated how we managed to solve these problems extending, optimizing and correcting these past numerical strategies.
    
    %% PHYSICAL PROBLEM
    \subsection{The physical problem}
    We intend to present the physical meaning of the Bidomain equations only briefly since it has already been widely shown in the previous project (\citeauthor{marta}). For a more complete explanation, we instead refer to \cite{acta}.\\
    The mechanical contraction and expansion of human heart has its origin in the \emph{electrical activation} of the cardiac cells. At every heart-beat, myocyties are activated and deactivated following a characteristic electrical cycle (fig \ref{potential_cycle}). 
    
    
    \begin{figure}[h]
    \begin{center}
    \includegraphics[width = 7cm]{./potential_cycle.png}
    \caption{Membrane potential in function of time (one cardiac cycle)}
    \label{potential_cycle}
    \end{center}
    \end{figure}
    
    \noindent The cell is initially at rest ($-90mV$, step 4). At a certain point, its potential increases rapidly ($\approx2ms$) and reaches the value of $+20mV$: the cell is activated. Later, a plateau near $0mV$ is observed and then a slow repolarization to the initial potential. \\
    From a microscopical point of view, we could study the dynamics acting in each single cell (as a consequence of the passage of chemical ions through specific channels, e.g. $Ca2+,Na+,K+$). From a macroscopical point of view, instead, one can observe it as a continuous electrical diffusion over the entire cardiac surface. Even if this consists in a very rapid phenomenon, the study of such propagation could be very interesting in order, for instance, to detect diseases in sick patients.
    
    \subsection{Mathematical models}
    Starting from the circuit in figure \ref{electrical_circuit}, applying some general electromagnetism laws and some calculations, the Bidomain model has been formulated (see \parencite{acta} for more details and/or \parencite{colli_franzone} for the complete passages).
    
    \begin{figure}[h]
    	\begin{center}
    		\includegraphics[width = 7cm]{./electrical_circuit.png}
    		\caption{Simplified circuit to model the intracellular and extracellular potentials dynamics}
    		\label{electrical_circuit}
    	\end{center}
    \end{figure}
    
    \noindent The general formulation is then: \vspace{3mm}
    \begin{definition}[Bidomain model]
	\begin{equation*}
	\begin{cases}
	\chi_m C_m\pdv{V_m}{t} - \nabla \cdot (\Sigma_i \nabla \phi_i) + \chi_m I_{ion} = I_i^{ext}    & \text{in } \Omega_{mus} \cross (0,T]
	\\
	-\chi_m C_m\pdv{V_m}{t} - \nabla \cdot (\Sigma_e \nabla \phi_e) - \chi_m I_{ion} = -I_e^{ext}    & \text{in } \Omega_{mus} \cross (0,T]
	\end{cases}
	\end{equation*}
    \end{definition}
	\vspace{3mm}
	
	where:
	\begin{itemize}[label=\textendash]
		\item $\bm{\phi_i, \phi_e}$ are the \emph{Intracellular and Extracelllular Potentials} (unknowns)
		\item $V_m = \phi_i-\phi_e$ is the \emph{Trans-membrane Potential}
		\item $\chi_m,C_m$ are known positive constants
		\item $\Sigma_i, \Sigma_e$ are known positive definite tensors 
		\item $I_i^{ext},I_e^{ext}$ are applied currents
		\item $I_{ion}$ is the \emph{Ionic Current}
		\item $\Omega_{mus}$ is the cardiac domain (myocardium + endocardium + epicardium)
	\end{itemize}
    
    \vspace{4mm}
    \noindent Actually, this system is not complete since it misses boundary and initial conditions and a suitable model for $I_{ion}$. Initial conditions and Neumann boundary conditions for $\phi_i$ and $\phi_e$ are then imposed. For the definition of $I_{ion}$, instead, a \emph{reduced ionic model} is chosen, in particular the \emph{FitzHugh-Nagumo model}. Summing up:
    
    \begin{definition}[Bidomain + FitzHugh-Nagumo model with Neumann boundary conditions]\label{def1}
    	\begin{equation*}
    	\begin{cases}
    	\chi_m C_m\pdv{V_m}{t} - \nabla \cdot (\Sigma_i \nabla \phi_i) + \chi_m I_{ion}(V_m,w) = I_i^{ext}    & \text{in } \Omega_{mus} \cross (0,T]
    	\\
    	-\chi_m C_m\pdv{V_m}{t} - \nabla \cdot (\Sigma_e \nabla \phi_e) - \chi_m I_{ion}(V_m,w) = -I_e^{ext}    & \text{in } \Omega_{mus} \cross (0,T]
    	\\
    	I_{ion}(V_m,w)=kV_m(V_m-a)(V_m-1)+w & \text{in } \Omega_{mus} \cross (0,T]
    	\\
    	\pdv{w}{t} = \epsilon(V_m-\gamma w)  & \text{in } \Omega_{mus} \cross (0,T]
    	\\
    	\Sigma_i\nabla \phi_i \cdot n = b_i   & \text{on } \partial \Omega_{mus} \cross (0,T]
    	\\
    	\Sigma_e\nabla \phi_e \cdot n = b_e   & \text{on } \partial \Omega_{mus} \cross (0,T]
    	\\
    	\text{Initial conditions for } \phi_i,\phi_e, w & \text{in } \Omega_{mus}\cross\{t=0\}
    	\end{cases}
    	\end{equation*}
    \end{definition}
    \vspace{3mm}
    where:
    \begin{itemize}[label=\textendash]
    	\item $\bm{w}$ is the \emph{gating variable} (unknown)
    	\item $k,a,\epsilon,\gamma$ are known constants
    	\item $b_i,b_e$ are the boundary conditions data
    	\item $n$ is the outward normal vector
    \end{itemize}

    \vspace{4mm}
    \noindent From now on, the system of definition \ref{def1} will be the reference analytical problem for the development of numerical schemes.\\
    To conclude, there exist other famous and useful models, such as the \emph{Monodomain model}. But this is just a simplification of the Bidomain as in this case it is assumed that $\phi_i$ and $\phi_e$ are proportional. However, thanks to its simplicity, we often tested the code starting from the Monodomain implementation of the project \cite{andreotti} instead of analyzing directly the Bidomain.

    \subsection{A short discussion about the past works issue}
    As we have already introduced, our project initially aimed to continue and improve the work of a previous project (\parencite{marta}). \\
    Results obtained using unitary parameters, namely $\chi_m =\Sigma_i= \Sigma_e= C_m= k = \epsilon= \gamma= a=1$, were actually quite satisfactory. On the other hand, the choice of more realistic/experimental values for the parameters (that are often very big or very small) caused bad consequences to the accuracy of the schemes or even to their stability. In particular, we observed that the choice of $C_m \approx 10^{-2}$ highly compromised the stability of the numerical schemes (fact that was already noticed in \cite{andreotti}).
    This issue heavily limits the use of the \textsc{Matlab} code for research and/or experimental simulations as it guarantees convergence to the right solution only in few and non-realistic problems.  \\

    \noindent After a while, we realized that an inverted sign of the FitzHugh-Nagumo model formula occurred in \cite{acta}. \\

    \noindent This oversight was not only essential for the fidelity to the real phenomena but also crucial for the well-posedness of the problem.\\ 
    We could give two motivations to reinforce this last statement: first of all, if we consider the well known study of \citeauthor{bourgault} (\cite{bourgault}), the conditions required for the well-posedness of the Bidomain problem are not satisfied if the sign is inverted (neither for the existence, hypothesis H4, nor for the uniqueness).\\ Secondly, suppose to discretize the Bilinear problem in time and to treat the non-linear term semi-implicitly, as it will be done in section \ref{temporal_discretization}. Then, if we fix the time-step and if data from the previous time-step are given, we achieve a linear problem that can be easily switched into a weak formulation. From this analysis, we could observe that if $V_m<a$ or $V_m>1$ and $C_m$ is sufficiently small, the associated bilinear form is not coercive. Some consequences of this result occurred during simulations, as in figure \ref{sol_ill_posed}. \\
    
    \begin{figure}[h]
    	\begin{center}
    		\includegraphics[width = 11cm]{./sol_ill_posed.jpg}
    		\caption{Comparison between exact and computed solution with inverted FitzHugh-Nagumo model: huge errors arise only when $Vm<0$}
    		\label{sol_ill_posed}
    	\end{center}
    \end{figure}

    \noindent While first reason is completely analytical, the second one is less formal and with a mixed analytical-numerical approach because of the time discretization. As shown in \cite{andreotti}, instabilities could depend on the choice of temporal step too. This fact is confirmed since the positive term of previous coercivity constant is the ratio between $C_m$ and $\Delta t$. \\
    \noindent In conclusion, the issue of the past works \cite{andreotti}, \cite{marta} had not a simple numerical nature as expected, but it probably had an analytical origin too. In general, there are no guarantees for numerical stability nor for analytical well-posedness when we try to solve the inverted sign problem. However, it is useless to go deeper in this non-realistic problem. From now on, the sign modification will be taken for granted.

\newpage

    \section{Semi discretized numerical methods}
    \subsection{DG discrete formulation}
    Starting from the strong form in definition \ref{def1}, the next step is the achievement of a suitable Discontinuous Galerkin weak formulation. Full descriptions and justifications of all the terms are present in \cite{marta} . \\
    Let us introduce a triangulation $\tau_h$ over $\Omega$, with $\mathcal{F} _h=\mathcal{F} _h^I \cup \mathcal{F} _h^B$ set of the faces of the elements, which includes the internal and boundary faces respectively, and the DG space $V_h^k = \{v_h \in L^2(\Omega) : v_h|_\mathcal{K} \in \mathbb{P}^{k}(\mathcal{K})  \quad \forall \mathcal{K} \in \tau_h \}$, where $k$ is the degree of the piecewise continuous polynomial. Moreover, we define $N_h=dim(V_h^k)<\infty$. \vspace{4mm}
    \begin{problem}[DG weak formulation]
    For any $t\in[0,T]$ find $\Phi_h(t)=[\phi_i^h(t),\phi_e^h(t)]^T \in [V_h^k]^2$  and  $w_h(t) \in V_h^k$ such that: \\
    \begin{enumerate}
    \item 
    \begin{equation*}
    \begin{gathered}
    \sum_{K \in \tau_h} \int_K{ \chi_m C_m \pdv{V_m^h}{t} V_h dw}+a_i(\phi_i^h,v_h)+\sum_{K \in \tau_h} \int_K{ \chi_m k (V_m^h-1)(V_m^h-a) V_m^h v_h dw}+\\
    +\sum_{K \in \tau_h} \int_K{ \chi_m w_h v_h dw}=(I_i^{ext},v_h) \qquad \forall v_h \in V_h^p\\
    \end{gathered}
    \end{equation*}
    \item
    \begin{equation*}
    \begin{gathered}
    -\sum_{K \in \tau_h} \int_K{ \chi_m C_m \pdv{V_m^h}{t} V_h dw}+a_e(\phi_e^h,v_h)-\sum_{K \in \tau_h} \int_K{ \chi_m k (V_m^h-1)(V_m^h-a) V_m^h v_h dw}+\\
    -\sum_{K \in \tau_h} \int_K{ \chi_m w_h v_h dw}=(-I_e^{ext},v_h) \qquad \forall v_h \in V_h^p\\
    \end{gathered}
    \end{equation*}
    \item
    \begin{equation*}
    \sum_{K \in \tau_h} \int_K{\pdv{w_h}{t}v_h dw}=\sum_{k \in \tau_h} \int_K{\epsilon (V_m^h-\gamma w_h) v_h dw} \qquad \forall v_h \in V_h^p\\
    \end{equation*}
    \end{enumerate}
    \vspace{5mm}
    where:
    \vspace{3mm}
    \begin{equation*}
    \begin{aligned}
    \bullet& \quad a_k(\phi_k^h,v_h)=\sum_{K \in \tau_h} \int_K{(\Sigma_k \nabla_h \phi_k^h) \cdot \nabla_h v_h dw}-\sum_{F \in \mathcal{F}_h^I} \int_F { \{\{\Sigma_k \nabla_h \phi_k^h \}\} \cdot [[v_h]] d\sigma}+\\
    &-\delta \sum_{F \in \mathcal{F}_h^I} \int_F{ \{\{\Sigma_k \nabla_h v_h\}\} \cdot [[\phi_k^h]]d\sigma}+\sum_{F \in \mathcal{F}_h^I}\int_F {\gamma [[\phi_k^h]] \cdot [[v_h]] d\sigma} \qquad k=i,e\\
    \newline
    \bullet& \quad (I_i^{ext},v_h)=\sum_{K \in \tau_h} \int_K {I_i^{ext} v_h dw}+\int_{\partial w}{b_i v_h d\sigma}\\
    \newline
    \bullet& \quad (-I_e^{ext},v_h)=-\sum_{K \in \tau_h} \int_K {I_e^{ext} v_h dw}+\int_{\partial w}{b_e v_h d\sigma}
    \end{aligned}
    \end{equation*}
    
    \noindent Moreover, according to the choice of the coefficient $\delta$, we can define:
    \begin{itemize}
    \item $\delta=1$: Symmetric Interior Penalty method (SIP)
    \item $\delta=0$: Incomplete Interior Penalty method (IIP)
    \item $\delta=-1$: Non Symmetric Interior Penalty method (NIP) 
    \end{itemize}
     \vspace{2mm}
    \noindent And $\gamma := \alpha \frac{k^2}{h}$ ("Stabilization parameter"),$ \quad \alpha \in \mathbb{R}$ to be chosen high enough.
    \end{problem}

    \vspace{4mm}
    \subsection{Algebraic formulation}
    Taking $\{\varphi_j\}_{j=1}^{N_h}$ base of $V_h^k$, so that we can write
    \begin{equation*}
    \begin{gathered}
    \Phi_h(t) = \begin{bmatrix} \phi_i^h(t) \\ \phi_e^h(t) \end{bmatrix} = \begin {bmatrix}\sum_{j=1}^{N_h} \phi_{i,j}(t)\varphi_j \\ \sum_{j=1}^{N_h} \phi_{e,j}(t)\varphi_j \end{bmatrix}\\
    w_h(t) = \sum_{j=1}^{N_h}w_j(t)\varphi_j\\
    V_m^h(t)=\sum_{j=1}^{N_h} V_{m,j}(t) \phi_j=\sum_{j=1}^{N_h}(\phi_{i,j}(t)-\phi_{e,j}(t))\varphi_j
 \end{gathered}
 \end{equation*}
 Then, we introduce the matrices:
 \begin{equation}\label{matrices}
\begin{rcases}
(V_k)_{ij} &= \int_{w}\nabla\varphi_j \cdot \Sigma_k \nabla \varphi_i 
\\ (I_k^T)_{ij} &= \sum_{F \in F_h^I} \int_{F} \{\{\Sigma_k \nabla\varphi_j\}\} \cdot [[\varphi_i]] 
\\ (I_k)_{i,j} &= \sum_{F \in F_h^I} \int_{F} [[\varphi_j]] \cdot \{\{\Sigma_k \nabla \varphi_i\}\}
\\(S_k)_{i,j} &= \sum_{F \in F_h^I} \int_{F} \gamma_k[[\varphi_j]] \cdot [[\varphi_i]]
\end{rcases}
\begin{gathered}
\quad A_k = (V_k -I_k^T - \theta I_k + S_k)\\
k=i,e \\
\end{gathered}
\end{equation}
\begin{equation*}
\gamma_k\vert_F = (n_F^T \, \Sigma_k \, n_F) \,\gamma, \quad n_F \text{ outward normal vector of F}
\end{equation*}

\begin{equation}
A_i \qquad{\text{Intra-cellular stiffness matrix}}
\end{equation}
\begin{equation}
A_e \qquad{\text{Extra-cellular stiffness matrix}}
\end{equation}
\begin{equation}
M_{ij} = \sum_{K \in \tau_h}\int_K\varphi_j\varphi_i \qquad{\text{Mass matrix}}
\end{equation}
\begin{equation}
C(u_h)_{ij} =  \sum_{K \in \tau_h} \int_K \chi_m k(u_h-1)(u_h-a)\varphi_j\varphi_i \qquad{\text{Non-linear matrix}}
\end{equation}
\begin{equation}
F_k=\begin{bmatrix} F_{i,k} \\ F_{e,k} \end{bmatrix}=\begin{bmatrix} \int_{w} I_i^{ext}\varphi_k - \sum_{F \in F_h^B} \int_F b_i\varphi_k \\ - \int_{w} I_e^{ext}\varphi_k - \sum_{F \in F_h^B} \int_F b_e\varphi_k \end{bmatrix}
\end{equation}
\vspace{3mm} \\
Therefore, our semi-discrete algebraic formulation is: \vspace{3mm}
\begin{problem}[DG algebraic formulation]\label{algebraic}
Find $\Phi_h(t)=[\phi_i^h(t),\phi_e^h(t)]^T \in [V_h^k]^2$ and $w_h(t) \in V_h^k$ for any $t \in (0;T]$ such that:
\begin{equation*}
\begin{cases}
\chi_m Cm M \dot{V_m^h}+A_i \phi_i^h+C(V_m^h) V_m^h+\chi_m M w_h=F_i^h \vspace{2mm} \\ 
-\chi_m Cm M \dot{V_m^h}+A_e \phi_e^h-C(V_m^h) V_m^h-\chi_m M w_h=F_e^h \vspace{2mm} \\ 
M \dot{w}_h(t)=\epsilon M (V_m^h(t)-\gamma w_h(t))
\end{cases}
\end{equation*}
\end{problem}
 \vspace{5mm}
 \noindent An alternative and more compact version with block matrices: \vspace{3mm}
 \begin{problem}[DG algebraic formulation - 2] \label{block_matrix}
 Find $\Phi_h(t)=[\phi_i^h(t),\phi_e^h(t)]^T \in [V_h^k]^2$ and $w_h(t) \in V_h^k$ for any $t \in (0;T]$ such that:
 \begin{enumerate}[label=\Roman*]
 \item
 \begin{equation*}
 \begin{gathered}
 \chi_mC_m \begin{bmatrix}M &-M \\ -M & M \end{bmatrix}
	\begin{bmatrix}\bm{\dot{\phi}_i^h(t)} \\ \bm{\dot{\phi}_e^h(t)} \end{bmatrix}
	 + \begin{bmatrix}A_i & 0 \\ 0 & A_e \end{bmatrix}
	 \begin{bmatrix}\bm{\phi_i^h(t)} \\ \bm{\phi_e^h(t)} \end{bmatrix} +\\
	   \begin{bmatrix}C(V_m^h) & -C(V_m^h) \\ -C(V_m^h) & C(V_m^h) \end{bmatrix} 
	   \begin{bmatrix} \bm{\phi_i^h(t)} \\ \bm{\phi_e^h(t)}  \end{bmatrix} 
	   +\chi_m \begin{bmatrix}M & 0 \\ 0 & -M \end{bmatrix} 
	   	\begin{bmatrix}w_h(t) \\ w_h(t) \end{bmatrix} = 
	   	\begin{bmatrix} F_i^h \\ F_e^h\end{bmatrix}
 \end{gathered}
 \end{equation*}
 \item
 \begin{equation*}
	   \dot{w}_h(t)=\epsilon (V_m^h(t)-\gamma w_h(t))
\end{equation*}
\end{enumerate}
\end{problem}
\vspace{10mm}
\section{Dubiner Basis}
    \subsection{Analytical aspects}\label{analytical_aspects}
    So far, we have described a general semi-discrete discontinuous formulation without examining which basis to use to generate the $V_h^k$ space. Usually, the common choice consists in the classical hat functions from FEM, even if they need to be modified in order to be used in a discontinuous context. It is also one of the simplest choices, for this reason our provided code was initially implemented with this basis. However, the very novelty of this study is the adoption of a new kind of basis, completely different from the previous and commonly known as "\emph{Dubiner Basis}" \cite{dubiner}. \\
    How we will soon see, the peculiarity of this family of functions is that it consists of orthogonal polynomials defined on the reference triangle
    \begin{equation}
    \hat{K}=\{ (\xi, \eta) : \xi, \eta \ge 0,	\xi+\eta \le 1 \}
    \end{equation}
    and not on the reference square
    \begin{equation}
    \quad \quad \hat{Q}=\{ (a, b) : -1 \le a \le 1, -1 \le b \le 1 \}
    \end{equation}
    Formally, if we consider the transformation from $\hat{Q}$ to $\hat{K}$
    \begin{equation}\label{transformation_formula}
    \xi:=\frac{(1+a)(1-b)}{4},  \eta:=\frac{(1+b)}{2}
    \end{equation}
    
    \begin{figure}[h]
    \begin{center}
    \includegraphics[width = 7cm]{./transformation.png}
    	\caption{Transformation between the reference square to the reference triangle}
    	\label{transformation}
    \end{center}
    \end{figure}
    
    \noindent the Dubiner basis is the transformation of a suitable basis initially defined on the reference square. This initial basis is simply obtained with a two dimensional modified tensor product of the Jacobi polynomials on the interval $(-1,1)$. \\

    \begin{definition}[Jacobi polynomials]
    The Jacobi polynomials of coefficients $\alpha,\beta \in \mathbb{R}$ evaluated in $z\in (-1,1)$ are:
    \begin{itemize}[label=\textendash]
    \item $n=0$
    \begin{equation}
    J_0^{\alpha,\beta}(z)=1
    \end{equation}
    \item $n=1$
    \begin{equation}
    J_1^{\alpha,\beta}(z)=\frac{1}{2}(\alpha-\beta+(\alpha+\beta+2)\cdot z);
    \end{equation}
    \item $n\ge2$
    \newline
    \begin{equation}
    \begin{gathered}
    \begin{aligned}
    J_n^{\alpha,\beta}(z)=\sum_{k=2}^{n} \Big[&\frac{(2k+\alpha+\beta-1)(\alpha^{2}-\beta^{2})}{2k(k+\alpha+\beta)(2k+\alpha+\beta-2)}+ \\ &\frac{(2k+\alpha+\beta-2)(2k+\alpha+\beta-1)(2k+\alpha \beta)}{2k(k+\alpha+\beta)(2k+\alpha+\beta-2)} J_{k-1}^{\alpha,\beta}(z) +
    \\-&\frac{2(k+\alpha-1)(k+\beta-1)(2k+\alpha+\beta)}{2k(k+\alpha+\beta)(2k+\alpha+\beta-2)} J_{k-2}^{\alpha,\beta}(z) \Big]
    \end{aligned}
    \end{gathered}
    \end{equation}
    \end{itemize}
    \end{definition}
    \vspace{5mm}
    \noindent The main property of these polynomials is:
    \begin{prop}
    $\{J_i^{\alpha,\beta}, \, i=0,1,2 \dots\}$ is orthogonal with respect to the Jacobi weight $w(x)=(1-x)^\alpha(1+x)^\beta$:
    \begin{equation}
    \int_{-1}^{1}{(1-x)^\alpha(1+x)^\beta J_m^{\alpha,\beta} J_q^{\alpha,\beta}(x)dx}=\frac{2}{2m+1} \delta_{mq} 
    \end{equation}
    \end{prop}
    
    \noindent We can now define explicitly the Dubiner basis.
    \begin{definition}[Dubiner Basis] \label{dubiner}
    The Dubiner basis that generates the space $\mathbb{P}^p(\hat{K})$ of the polynomials of degree $p$ over the reference triangle is the set of functions:
    \begin{equation}
    \begin{split}
    \\
    & \quad \quad\quad  \quad \phi_{ij}: \hat{K} \rightarrow \mathbb{R} \\ \\
    \phi_{ij}(\xi,\eta) :&= c_{ij}(1-b)^j J_i^{0,0}(a) J_j^{2i+1,0}(b)=
    \\&=c_{ij} 2^j (1-\eta)^j J_i^{0,0}(\frac{2\xi}{1-\eta}-1) J_j^{2i+1,0} (2\eta-1) 
    \end{split}
    \end{equation}
    \vspace{3mm} \\
    for $i,j=1,\dots,p$ and $i+j \le p$, where
    \begin{equation}
    c_{ij} := \sqrt{\frac{2(2i+1)(i+j+1)}{4^i}}
    \end{equation}
    and $J_i^{\alpha,\beta}(\cdot)$ is the i-th Jacobi polynomial
    \end{definition}
    
    \vspace{5mm}
    \noindent As we have anticipated \cite{sherwin}
    \begin{prop}\label{l2_ortho}
    The Dubiner basis is orthonormal in $L^2(\hat{K})$ $\forall p$:
    \begin{equation}
    \int_{\hat{K}}{\phi_{ij}(\xi,\eta)\phi_{mq}(\xi,\eta) d\xi d\eta}=\delta_{im}\delta_{jq}
    \end{equation}
    \end{prop}
    \vspace{4mm}
    \noindent As a consequence, after we successfully implemented the code with Dubiner basis and computed the matrices, we obtained a diagonal mass matrix (figure \ref{mass})
    
    \begin{figure}[ht]
    \begin{center}
    \includegraphics[width = 7cm]{./mass_dubiner.jpg}
    	\caption{Non-zero elements in the mass matrix when adopting Dubiner basis}
    	\label{mass}
    \end{center}
    \end{figure}

    \noindent It is noteworthy to point out that transformation \ref{transformation_formula} is bijective, it can be inverted but it needs some care. The natural inverse would be:
    
    \begin{equation}
    a = \frac{2\xi}{1-\eta}-1 \quad \quad b = 2\eta-1
    \end{equation}
    \vspace{2mm} \\
    \noindent that has already been used for the definition \ref{dubiner}. However, it is not defined for $\eta=1$, that means for the sole point $(0,1)$ of the reference triangle. To solve this issue, it is enough to prolong the function with continuity to this special point. For the code implementation, it is suggested avoiding evaluations in the exact point or adding an \emph{if} condition. We opted for the second solution. \\
    
    \noindent In general, the orthogonality property implies some good numerical properties, not only the diagonalization of the mass matrix. For instance, in \cite{antonietti} interesting bounds for the conditional number can be viewed. For this reason, we opted for this choice aiming to improve the previous results, at least from the space discretization side. \\
    
    \noindent On the other hand, there are also some difficulties arising when one chooses to abandon the familiar FEM basis. First of all, the coefficients of a discretized function has only \emph{modal} meaning and they no more represent the \emph{nodal} values of the function itself. This fact needs some extra work when one needs to switch from the continuous functions to the discretized functions and viceversa, as it will be shown in the paragraph \ref{subsection_implementation}. Secondly, one can notice that these functions are not boundary conditions friendly. What we mean is that, if compared to FEM basis, they have no particular properties on the boundary to let easily impose homogeneous boundary conditions. Thus, they should be again transformed, this time in a \emph{boundary adapted} form. We address to \cite{napde} for a short description of this procedure. Fortunately, we do not need to set this transformation as in the discontinuous formulation boundary conditions (both Dirichlet and Neumann) are imposed only weakly. It means that the boundary conditions' choice does not imply the choice of the vectorial space as in continuous Galerkin. The discretized space is always the same, only some terms in the weak formulation have in case of need to be changed. For this reason, the match of Discontinuous Galerkin and Dubiner Basis results to be particularly successful. \\
    
    \noindent To conclude, we refer to \cite{sherwin} for the transformation and the definition of the Dubiner Basis with tetrahedra, i.e. in dimension $n=3$.\\ \\
    
    \subsection{Implementation}\label{subsection_implementation}
    Our \textsc{Matlab} code let the user to select which basis to adopt (FEM or Dubiner) and the order of polynomials until the order 3. We chose to call $D_1,D_2,D_3$ these 3 families of basis functions, thanks to the similarity to the $P_1,P_2,P_3$ finite element basis.\\
    The starting point was the implementation of some functions to evaluate the Dubiner basis functions and their gradients in the quadrature points. We omit the full code as it is not particular interesting: it barely follows the definitions of section \ref{analytical_aspects} with the addition of some technicalities. These mentioned scripts are: \texttt{eval\_jacobi\_polynomial.m}, \texttt{basis\_legendre\_dubiner.m}, \texttt{evalshape\_tria\_dubiner.m}.\\
    
    \noindent Moreover, some conditional statements and some extra methods (as \texttt{matrix2D\_dubiner.m}) were added to let the user easily switch from one basis to another (simply and once via \texttt{dati.m}). \\
    
    \noindent More interesting are instead the scripts \texttt{dubiner\_to\_fem.m} and \texttt{fem\_to\_dubiner.m}, used to convert the Dubiner modal coefficients of the vector solution to the nodal values of the approximated function and viceversa. For this reason, they deserve some further explanations.
\subsection{Switch from the modal coefficients to the nodal values}
One of the many advantages of the FEM basis is that there exists a bijection between the basis functions and some particular spacial points in such a way that the evaluation of a basis function in one of these points is equal to 1 only if that point is the one associated to the function, 0 otherwise:
	\begin{equation} \label{ref1}
	\psi_i(x_j)=\delta_{ij}
	\end{equation}
	Obviously, this property does not hold when we work with Dubiner basis. Indeed, these functions are not normalized on the mesh edges and they neither have an associated mesh point. This implies that the Dubiner coefficients of a function $u\in V_h^p$ are not the evaluation over these points of the discretized function itself. They have a completely different meaning, they are now \emph{modal} values instead of being \emph{nodal}.
	For this reason we introduced two new functions that best transform the coefficients of the solution w.r.t. FEM basis to the coefficients w.r.t. Dubiner basis and viceversa.\vspace{5mm}
	
	\noindent Consider an element $\mathcal{K}\in \tau_h$ and $\{\psi_{i}\}_{i=1}^{p}$,$\{\phi_{j}\}_{j=1}^{q}$ as, respectively, the set of FEM functions and the set of Dubiner functions with support in $\mathcal{K}$. In addition, consider as $\{\hat{u}_i\}_{i=1}^p$,$\{\tilde{u}_j\}_{j=1}^q$ as, respectively, the FEM and Dubiner coefficients of the solution. \vspace{5mm}
	
	\subsubsection{\texttt{dubiner\_to\_fem.m}}
	\noindent Let us start from the transformation to the FEM coefficients. We now exploit the property \ref{ref1}, i.e. the coefficient $\hat{u}_i$ is nothing else but the evaluation of $u_h$ on the i-th mesh point, then: 
	\begin{equation} \label{ref3}
	\hat{u}_i = \sum_{j=1}^q \tilde{u}_j\phi_j(x_i)
	\end{equation}
	where $x_i$ is the point associated to the $\psi_i$ basis function. \vspace{5mm} \\
	This formula has been implemented in the following lines from \texttt{dubiner\_to\_fem.m} script. \\
	\begin{lstlisting}[language=Matlab,basicstyle=\small, numbers=left, numberstyle=\tiny,  name = dubiner_to_fem.m, frame=single]
function [u0] = dubiner_to_fem (uh, femregion, Data)  
	
...
...
...
	
u0 = zeros(femregion.ndof,1);
	
% loop over all the elements
for ie = 1:femregion.ne
	
   % to get the global indexes for the nodes of ie 
   nln = femregion.nln;
   index = (ie-1)*nln*ones(nln,1) + [1:nln]';
	
   for i = 1 : nln
      for j = 1 : nln
         u0(index(i)) = u0(index(i)) +  uh(index(j))*phi(1,i,j);
      end
   end
end
	\end{lstlisting} \vspace{4mm}
	\subsubsection{\texttt{fem\_to\_dubiner.m}}
	\noindent Instead, to compute the coefficients conversely, we need to exploit the fact that the Dubiner Basis are $L^2$-orthonormal (proposition \ref{l2_ortho}). We then need to compute a $L^2$ scalar product between the FEM discretized function and each Dubiner basis function. That means:
	\begin{equation}\label{ref4}
	\tilde{u}_j = \int_\mathcal{K} u_h(x) \phi_j(x) \,dx = \int_{\mathcal{K}} \sum_{i=1}^p \hat{u}_i\psi_i(x) \phi_j(x) \,dx = \sum_{i=1}^p \Big(\int_{\mathcal{K}}\psi_i(x)\phi_j(x)\,dx \Big) \hat{u}_i
	\end{equation}
	\vspace{2mm} \\
	\noindent This slightly more difficult formula has been reproduced in \texttt{fem\_to\_dubiner.m} using \emph{Gauss-Legendre-Lobatto} quadrature formulas. The main steps are the following: \\
	\begin{lstlisting}[language=Matlab,basicstyle=\small, numbers=left, numberstyle=\tiny,  name = fem_to_dubiner.m, frame=single]
function [u0] = fem_to_dubiner (uh, femregion, Data)
	
...
...
...
	
u0 = zeros(femregion.ndof,1);
	
% loop over all the elements
for ie = 1:femregion.ne
	
   % to get the global indexes for the nodes of ie 
   nln = femregion.nln;
   index = (ie-1)*nln*ones(nln,1) + [1:nln]';
   % loop over local degrees of freedom
   for i = 1 : nln
      % loop over 2D quadrature points
      for k = 1:length(w_2D) 
         uh_eval_k = 0;
         % loop to evaluate uh in a quadrature point
         for j = 1 : nln
            uh_eval_k = uh_eval_k + uh(index(j))*phi_fem(1,k,j);
         end
         u0(index(i))=u0(index(i))+uh_eval_k*phi_dub(1,k,i).*w_2D(k);
      end
   end    
end
	\end{lstlisting}
	
	\vspace{5mm}
	\subsubsection{Final remarks} 
	\noindent If the Dubiner functions are chosen as Galerkin basis, both the transformations are needed for the code implementation. Formula \ref{ref3} is needed to plot and compute errors after the resolution of the system (otherwise solely Dubiner coefficients are useless). Formula \ref{ref4} is instead needed to convert the FEM initial data $u_0$ into a vector of Dubiner coefficients before the resolution of the system.\\
	
	\noindent In order to be rigorous, but also for the sake of simplicity, these transformations are implemented only from $P_n$ to $D_n$, $n=1,2,3$ and viceversa. With this choice, the two basis generate the same space $V_h^n$ and then the transformation infers only on the coefficients and not on the function's properties. Otherwise, decreasing $n$ would mean to lose significant information, while increasing $n$ does not substantially improve the quality of the solution as it initially belonged to a lower order space. Moreover, choosing the same degree for P and D implies several semplifications, for instance the same number of local nodes ($nln$). For this reason, both $p$ and $q$ are actually replaced with $nln$ in the code.\\
	

\newpage 
\section{Temporal discretization}\label{temporal_discretization}
So far, we have just studied the space discretization while a temporal discretization is still needed to totally discretize the Bidomain time-dependent problem. Thus, we divide the interval (0,T] into K subintervals $(t^k,t^k+1]$ of length $\Delta t$ such that $t^k=k \Delta t \quad \forall k=0,\cdots,K-1$, we then assume that $V_m^k\approx V_m(t^k)$. \\
We have developed, implemented and tested 3 different temporal strategies that we will refer to as: \emph{semi-implicit}, \emph{Godunov operator-splitting} and \emph{quasi-implicit operator-splitting}. 
\subsection{Semi-implicit method}
One of the most famous and used temporal scheme for a non-linear problem such as the Bidomain is certainly the \emph{Semi-Implicit} scheme \cite{acta}. The basic idea is to treat most of the terms implicitly while in fact treating the non-linear term semi-implicitly. Since the non-linear is cubic, the best choice is to treat only one of these $V_m$ terms implicitly, i.e.:
\begin{equation*}
I_{ion}^{k+1}=k(V_m^k-a)(V_m^k-1)V_m^{k+1}+w^{k+1}
\end{equation*}
at each time-step $k$ (different from the $k$ parameter). 
Moreover, the gating variable ODE is treated implicitly with the exception of the term $V_m$:
\begin{equation*}
M \frac{w^{k+1}-w^k}{\Delta t}=\epsilon M (V_m^k-\gamma w^{k+1})
\end{equation*}
\vspace{5mm} \\
Therefore, we can transform the semi-discrete problem \ref{block_matrix} into:
\begin{equation}
 \begin{gathered}
 \chi_mC_m \begin{bmatrix}M &-M \\ -M & M \end{bmatrix}
	\begin{bmatrix}\bm{\frac{\phi_i^{k+1}-\phi_i^{k}}{\Delta t}} \\ \bm{\frac{\phi_e^{k+1}-\phi_e^{k}}{\Delta t}}  \end{bmatrix}
	 + \begin{bmatrix}A_i & 0 \\ 0 & A_e \end{bmatrix}
	 \begin{bmatrix}\bm{\phi_i^{k+1}} \\ \bm{\phi_e^{k+1}} \end{bmatrix} +\\
	   \begin{bmatrix}C(V_m^h) & -C(V_m^h) \\ -C(V_m^h) & C(V_m^h) \end{bmatrix} 
	   \begin{bmatrix} \bm{\phi_i^{k+1}} \\ \bm{\phi_e^{k+1}}  \end{bmatrix} 
	   +\chi_m \begin{bmatrix}M & 0 \\ 0 & -M \end{bmatrix} 
	   	\begin{bmatrix}w^{k+1} \\ w^{k+1} \end{bmatrix} = 
	   	\begin{bmatrix} F_i^{k+1} \\ F_e^{k+1}\end{bmatrix}\\
	   M \frac{w^{k+1}-w^k}{\Delta t}=\epsilon M (V_m^k-\gamma w^{k+1})
\end{gathered}
\end{equation}\\
We remind that $V_m^k=\phi_i^k-\phi_e^k$. Separating known and unknown terms, we obtain: \vspace{3mm}
\begin{equation}
\begin{gathered}
\left( \frac{\chi_m C_m}{\Delta t} \begin{bmatrix} M & -M \\ -M & M \end{bmatrix} + \begin{bmatrix} A_i & 0 \\ 0 & A_e \end{bmatrix} + 
\begin{bmatrix}
C(V_m^k) & -C(V_m^k) \\ -C(V_m^k) & C(V_m^k)
\end{bmatrix}\right)
\begin{bmatrix} \bm{\phi_i^{k+1}} \\ \bm{\phi_e^{k+1}} \end{bmatrix} = 
\\
\begin{bmatrix} F_i^{k+1} \\ F_e^{k+1} \end{bmatrix} 
- \chi_m \begin{bmatrix}M & 0 \\ 0 & -M \end{bmatrix}
\begin{bmatrix} w^{k+1} \\ w^{k+1} \end{bmatrix}
+ \frac{\chi_m C_m}{\Delta t} \begin{bmatrix}M & 0 \\ 0 & -M\end{bmatrix}
\begin{bmatrix} V_m^{k} \\ V_m^{k} \end{bmatrix} \\
(\frac{1}{\Delta t}+\epsilon \gamma)M w^{k+1}=\epsilon M V_m^k+\frac{M}{\Delta t} w^k
\end{gathered}
\end{equation}

\vspace{5mm}
\noindent If we define:
\begin{itemize}
\item $B=\frac{\chi_m C_m}{\Delta t} \begin{bmatrix} M &-M \\-M & M \end{bmatrix}+\begin{bmatrix} A_i & 0 \\ 0 & A_e \end{bmatrix}$
\item $C_{nl}(V_m^k)=\begin{bmatrix}C(V_m^h) & -C(V_m^h) \\ -C(V_m^h) & C(V_m^h) \end{bmatrix}$
\item $r^{k+1}=\begin{bmatrix} F_i^{k+1} \\ F_e^{k+1}\end{bmatrix}-\chi_m \begin{bmatrix}M & 0 \\ 0 & -M \end{bmatrix} \begin{bmatrix}w^{k+1} \\ w^{k+1} \end{bmatrix}+\frac{\chi_m C_m}{\Delta t} \begin{bmatrix} M &-M \\-M & M \end{bmatrix} \begin{bmatrix} \phi_i^k \\ \phi_e^k \end{bmatrix}$
\end{itemize} \vspace{5mm}
we get the system in the final form:  \\
\begin{problem}[Semi-implicit discretized system]
Find $\Phi^{k+1}=[\phi_i^{k+1} \phi_e^{k+1}]^T$ and $w^{k+1}$ $\forall k=0,\cdots,T-1$ such that:
\begin{equation*}
\begin{cases}
(\frac{1}{\Delta t}+\epsilon \gamma)M w^{k+1}=\epsilon M V_m^k+\frac{M}{\Delta t} w^k \\
(B+C_{nl}(\Phi^k)) \Phi^{k+1}=r^{k+1}
\end{cases}
\end{equation*} 
\end{problem}\vspace{4mm}

\subsubsection{Implementation}
\begin{lstlisting}[language=Matlab,basicstyle=\small, numbers=left, numberstyle=\tiny,  name = main2D.m (semi-implicit), frame=single]
MASS = [M -M; -M M];
ZERO = sparse(length(M), length(M));
MASS_W = [M ZERO; ZERO -M];
STIFFNESS = [Ai ZERO; ZERO  Ae];

for t=dt:dt:T

   w1 = 1/(1+epsilon*gamma*dt)*(w0+epsilon*dt*Vm0);
   w1=cat(1,w1, w1);
   Vm0 = cat(1,Vm0,Vm0);

   fi = assemble_rhs_i(femregion,neighbour,Data,t);
   fe = assemble_rhs_e(femregion,neighbour,Data,t);
   f1 = cat(1, fi, fe);

   [C] = assemble_nonlinear(femregion,Data,Vm0);
   NONLIN = [C -C; -C C];

   r = f1 + ChiM*Cm/dt * MASS_W * Vm0 - ChiM * MASS_W *w1;

   B=ChiM*Cm/dt * MASS + (STIFFNESS + NONLIN);

   u1 = B \ r;

   f0 = f1;
   Vm0 = u1(1:ll)-u1(ll+1:end);
   u0 = u1;
   w0 = w1(1:ll);
end
\end{lstlisting}
\vspace{4mm}
\subsection{Godunov operator-splitting}
The main feature of a general operator-splitting method is the division of the problem into two different problems to be solved sequentially. This is possible and justified when the original functional operator $L$ is splitted into 2 different operators such that $L(u)=L_1(u)+L_2(u)$.
Two operator-splitting methods have been implemented, the first is of \emph{Godunov} type and a detailed study together with its properties can be found in \cite{spiteri}. The formulation is: \\ \\
\begin{center}Find $\hat{V}_m^{k+1},\phi_i^{k+1},\phi_e^{k+1},w^{k+1}$ such that:\end{center}
\begin{enumerate}[label = \Roman*]
\item
\begin{equation*}
\begin{cases}
\chi_m C_m M\frac{\hat{V}_m^{k+1} - V_m^k}{\Delta t} + C(V_m^k)V_m^k +\chi_mMw^k = 0\\
\frac{w^{k+1}-w^k}{\Delta t} = \epsilon (V_m^k - \gamma w^k)
\end{cases}
\end{equation*}
\item
\begin{equation*}
\begin{cases}
\chi_m C_m M\frac{V_m^{k+1} - \hat{V}_m^{k+1}}{\Delta t} + A_i\phi_i^{k+1} = F_i^{k+1} \\
-\chi_m C_m M\frac{V_m^{k+1} - \hat{V}_m^{k+1}}{\Delta t} + A_e\phi_e^{k+1} = F_e^{k+1}
\end{cases}
\end{equation*}
\end{enumerate}\vspace{3mm}
\noindent Putting in a unique system:
\begin{equation}\label{Godunov}
\begin{cases}
\chi_m C_m M\frac{V_m^{k+1} - V_m^k}{\Delta t} + C(V_m^k)V_m^k +\chi_mMw^k + A_i \phi_i^{k+1}= F_i^{k+1} \\
\chi_m C_m M\frac{V_m^{k+1} - V_m^k}{\Delta t} + C(V_m^k)V_m^k +\chi_mMw^k - A_e \Phi_e^{k+1}= -F_e^{k+1} \\
w^{k+1} = (1-\epsilon \gamma \Delta t) w^k + \epsilon \Delta t V_m^k
\end{cases}
\end{equation} \vspace{3mm} \\
The equations in the system \ref{Godunov} can be rewritten as:
\begin{equation*}
\begin{cases}
\left( \frac{\chi_m C_m}{\Delta t} M + A_i \right ) \phi_i^{k+1} - \frac{\chi_m C_m}{\Delta t} M \phi_e^{k+1} = F_i^{k+1} - \chi_m M w^k + \left( \frac{\chi_m C_m}{\Delta t} M- C(V_m^k)\right) V_m^k\\
\frac{\chi_m C_m}{\Delta t} M  \phi_i^{n+1} - \left(\frac{\chi_m C_m}{\Delta t} M + A_e \right) \phi_e^{k+1} =  -F_e^{k+1} - \chi_m M w^k + \left( \frac{\chi_m C_m}{\Delta t} M- C(V_m^k)\right) V_m^k \\
w^{k+1} = (1-\epsilon \gamma \Delta t) w^k + \epsilon \Delta tV_m^k
\end{cases}
\end{equation*}

\noindent And then, we obtain the final form:
\begin{problem}[Godunov operator-splitting discretized system] 
	Find $\Phi^{k+1}=[\phi_i^{k+1} \phi_e^{k+1}]^T$ and $w^{k+1} \quad \forall k=0, \cdots, K-1$ such that: 
\begin{equation*}
\begin{cases}
\begin{gathered}
\left(
\frac{\chi_m C_m}{\Delta t} \begin{bmatrix}M & -M \\ M & -M\end{bmatrix}
+ \begin{bmatrix} A_i & 0 \\ 0 & -A_e \end{bmatrix}
\right) \begin{bmatrix} \bm{\phi_i^{k+1}} \\ \bm{\phi_e^{k+1}}  \end{bmatrix} =
\begin{bmatrix} F_i^{k+1} \\ -F_e^{k+1} \end{bmatrix} + \\ -
\chi_m\begin{bmatrix} M & 0 \\ 0 & M \end{bmatrix} \begin{bmatrix} w^k \\ w^k \end{bmatrix} +
\left(\frac{\chi_mC_m}{\Delta t}\begin{bmatrix} M & 0 \\ 0 & M \end{bmatrix}
- \begin{bmatrix} C(V_m^k) & 0 \\ 0 & C(V_m^k)\end{bmatrix} 
\right) \begin{bmatrix} V_m^k \\ V_m^k \end {bmatrix}
\end{gathered} \\ \\
w^{k+1} = (1-\epsilon \gamma \Delta t) w^k + \epsilon \Delta tV_m^k
\end{cases}
\end{equation*}
\end{problem}\vspace{4mm}

\subsubsection{Implementation}
\begin{lstlisting}[language=Matlab,basicstyle=\small, numbers=left, numberstyle=\tiny,  name = main2D.m (Godunov operator-splitting), frame=single]
ZERO = sparse(ll,ll);
MASS = (ChiM*Cm/dt)*[M, -M; M -M];
MASSW = ChiM*[M, ZERO; ZERO, M];

for t=dt:dt:T

   fi = assemble_rhs_i(femregion,neighbour,Data,t);
   fe = assemble_rhs_e(femregion,neighbour,Data,t);
   f1 = cat(1, fi, -fe);

   [C] = assemble_nonlinear(femregion,Data,Vm0);

   w1 = (1 -epsilon*gamma*dt)*w0 + epsilon*dt*Vm0;
   B = MASS + [Ai, ZERO; ZERO, -Ae];
   r = -MASSW*[w0;w0] + ((Cm/dt)*MASSW - [C, ZERO; ZERO, C])
   *[Vm0;Vm0] + f1;

   Vm0 = u1(1:ll) - u1(ll+1:end);
   u0 = u1;
   w0 = w1;
end
\end{lstlisting}
\vspace{4mm}
\subsection{Quasi-implicit operator-splitting}
The aim of a quasi-implicit operator splitting is to treat implicitly all the terms except the cubic one. Even if it cannot be defined as a full implicit method, we hope to achieve more stability if compared to the previous Godunov-kind scheme. This time, the formulation turns to be: \newline

\begin{center} Find $\tilde{V}_m^{k+1}, \phi_i^{k+1}, \phi_e^{k+1},w^{k+1}$ such that: \end{center}
\begin{enumerate}[label = \Roman*]
\item
\begin{equation*}
\begin{cases}
\chi_m C_m M \frac{\tilde{V}_m^{k+1}-V_m^k}{\Delta t} +  C(V_m^k) V_m^{k+1} + \chi_m M w^{k+1}= 0\\
\frac{w^{k+1} - w^k}{\Delta t} = \epsilon (V_m^{k+1}-\gamma w^{k+1})
\end{cases}
\end{equation*}
\item
\begin{equation*}
\begin{cases}
\chi_m C_m M \frac{V_m^{k+1}-\tilde{V}_m^{k+1}}{\Delta t} + A_i \phi_i^{k+1}= F_i^{k+1}\\
- \chi_m C_m M \frac{V_m^{k+1}-\tilde{V}_m^{k+1}}{\Delta t} + A_e \phi_e^{k+1}= F_e^{k+1}
\end{cases}
\end{equation*}
\end{enumerate}
\vspace{3mm}
Putting into a unique system:
\begin{equation}\label{Quasi}
\begin{cases}
\chi_m C_m M \frac{V_m^{k+1}-V_m^{k}}{\Delta t} + C(V_m^k) V_m^{k+1} + \chi_m M w^{k+1} + A_i \phi_i ^{k+1} = F_i^{k+1} \\
\chi_m C_m M \frac{V_m^{k+1}-V_m^{k}}{\Delta t} +  C(V_m^k) V_m^{k+1} + \chi_m M w^{k+1} - A_e \phi_e ^{k+1} =  -F_e^{k+1} \\
\frac{w^{k+1}-w^{k}}{\Delta t} = \epsilon(V_m^{k+1}-\gamma w^{k+1})
\end{cases}
\end{equation} 
\vspace{14mm} \\
If we define:
\begin{itemize}
\item $Q_k = \frac{\chi_m C_m}{\Delta t}M + C(V_m^k) + \frac{\epsilon\chi_m \Delta t}{1 + \epsilon \gamma \Delta t} M$ 
\item $R_k = \frac{\chi_mC_m}{\Delta t}MV_m^k - \frac{\chi_m}{1+\epsilon\gamma\Delta t}M w^k$
\end{itemize}
\vspace{4mm}
the equations in system \ref{Quasi} can be written as:
\begin{enumerate}
\item
\begin{equation*}
\begin{gathered}
\chi_m C_m M \frac{	\phi_i^{k+1}-\phi_e^{k+1}-V_m^{k}}{\Delta t} +  C(V_m^k) (\phi_i^{k+1}-\phi_e^{k+1}) + \\
 +\chi_m M \left(\frac{w^k + \epsilon \Delta t (\phi_i^{k+1}-\phi_e^{k+1})}{1+\epsilon \gamma \Delta t}   \right)
+ A_i \phi_i ^{k+1} = F_i^{k+1} \\ \\
\Rightarrow \quad (Q_k + A_i) \phi_i^{k+1} - Q_k \phi_e^{k+1} =R_k +  F_i^{k+1}
\end{gathered}
\end{equation*}
\item
\begin{equation*}
\begin{gathered}
\chi_m C_m M \frac{	\phi_i^{k+1}-\phi_e^{k+1}-V_m^{k}}{\Delta t} + \cdot C(V_m^k) (\phi_i^{k+1}-\phi_e^{k+1}) +\\ 
+ \chi_m M \left(\frac{w^k + \epsilon \Delta t (\phi_i^{k+1}-\phi_e^{k+1})}{1+\epsilon \gamma \Delta t}   \right)
- A_e \phi_e ^{k+1} = -F_e^{k+1} \\ \\
\Rightarrow \quad Q_k \phi_i^{k+1} - (Q_k+A_e) \phi_e^{k+1} =R_k - F_e^{k+1}
\end{gathered}
\end{equation*}
\item 
\begin{equation*}
w^{k+1} = \frac{w^k + \epsilon \Delta t (\phi_i^{k+1}-\phi_e^{k+1})}{1+\epsilon \gamma \Delta t}
\end{equation*}
\end{enumerate}
\vspace{4mm}
The final system becomes: \\
\begin{problem} [Quasi-implicit operator-splitting discretized system]
Find $\Phi^{k+1}=[\phi_i^{k+1} \phi_e^{k+1}]^T$ and $w^{k+1} \quad \forall k=0, \cdots, K-1$ such that:
\begin{equation*}
\quad
\begin{cases}
\left(
\begin{bmatrix} Q_k & -Q_k \\ Q_k & -Q_k \end{bmatrix} + 
\begin{bmatrix} A_i & 0 \\ 0 & -A_e\end{bmatrix}
\right)
\begin{bmatrix}
\bm{\phi_i^{k+1}} \\ \bm{\phi_e^{k+1}}
\end{bmatrix}
= \begin{bmatrix} R_k \\ R_k \end{bmatrix} + \begin{bmatrix} F_i^{k+1} \\  -F_e^{k+1}\end{bmatrix} \\ \\
w^{k+1} = \frac{\displaystyle w^k + \epsilon \Delta t (\phi_i^{k+1}-\phi_e^{k+1})}{\displaystyle 1+\epsilon \gamma \Delta t}
\end{cases}
\end{equation*}
\end{problem}
\vspace{4mm}
\subsubsection{Implementation}
\begin{lstlisting}[language=Matlab,basicstyle=\small, numbers=left, numberstyle=\tiny,  name = main2D.m (quasi-implicit operator-splitting), frame=single]
ZERO = sparse(ll,ll);
        
for t=dt:dt:T
        
   [C] = assemble_nonlinear(femregion,Data,Vm0);
   Q  = (ChiM*Cm/dt)*M + C - (epsilon*ChiM*dt)/(1+epsilon*gamma*dt)*M;
   R  = (ChiM*Cm/dt)*M*Vm0 - (ChiM)/(1+epsilon*gamma*dt)*M*w0;
    
   fi = assemble_rhs_i(femregion,neighbour,Data,t);
   fe = assemble_rhs_e(femregion,neighbour,Data,t);
   f1 = cat(1, fi, -fe);
    
   B = [Q, -Q; Q, -Q] + [Ai, ZERO; ZERO, -Ae];
   r = [R;R] + f1;
        
   u1 = B \ r; 
        
   Vm1 = u1(1:ll)-u1(ll+1:end);

   w1 = (w0 + epsilon*dt*Vm1)/(1+epsilon*gamma*dt);
    
   f0 = f1;
   Vm0 = u1(1:ll) - u1(ll+1:end);
   u0 = u1;
   w0 = w1;
end
\end{lstlisting}
\newpage
\section{About the potentials uniqueness}
\subsection{Analytical concepts}
\noindent Observing for a moment the Bidomain problem analytical formulation (definition \ref{def1}), we can immediately realize that the intracellular and the extracellular potentials appear only through their difference $V_m$ or their gradient. This means that there cannot be uniqueness for the two functions. Namely: \vspace{2mm}
\begin{equation}\label{phi_uniqueness}
\begin{gathered}
\phi_i,\phi_e \text{ classical solutions of Bidomain} \Rightarrow \phi_i+\varphi,\,\phi_e+\varphi \text{ are solutions as well } \\
 \forall \varphi: [0,T] \rightarrow \mathbb{R} \quad \text{ sufficiently regular}
\end{gathered}
\end{equation}

\vspace{4mm}
\noindent However, this fact should not surprise nor confuse the reader. First of all, we remind again that in \cite{bourgault} and \cite{colli_franzone} there are proofs for the $V_m$ and $w$ uniqueness, then this is taken for granted. Secondly, this statement reflects the physical intuition of the problem: cellular dynamics is not involved by potentials exact values but instead from their difference, in addition a potential value is nonsense if a convention value to compare it with has not been set. The dependence on time can be interpreted as follows: if, at any time instant, we change the conventional potential value, the dynamics of the problem does not change. \\

\noindent Moreover, we can give this simple result to show that the solutions of the form of equation \ref{phi_uniqueness} are also the only admissible:\\
\begin{theo}
	For the Bidomain problem coupled with Fitzhugh-Nagumo model with Neumann boundary conditions (definition \ref{def1}), the $\phi_i,\phi_e$ solutions are unique minus a constant depending only on time. 
\end{theo}

\begin{proof}
	\vspace{2mm} We remind that existence and uniqueness for $V_m$ and $w$ have already been proved in \cite{bourgault}. 
	Suppose now there exist two couples $(\phi_i^1,\phi_e^1)$,$(\phi_i^2,\phi_e^2)$ of potentials solutions of the Bidomain problem. If $V_m$ uniqueness holds, there must exist a unique value of $V_m$ such that:
	\begin{equation*}
	\phi_i^1-\phi_e^1 = \phi_i^2-\phi_e^2 = V_m
	\end{equation*}\\
	Then, we define a function $\varphi:\Omega \cross [0,T] \rightarrow \mathbb{R}$ as:
	\begin{equation*}
	\varphi := \phi_i^1-\phi_i^2 = \phi_e^1-\phi_e^2
	\end{equation*}\\
	If we consider the starting problem in definition \ref{def1}, the following equations must hold:
	\begin{equation*}
	\begin{cases}
	\chi_m C_m\pdv{V_m}{t} - \nabla \cdot (\Sigma_i \nabla \phi_i^1) + \chi_m I_{ion}(V_m,w) = I_i^{ext}    & \text{in } \Omega_{mus} \cross (0,T]
	\\
	\chi_m C_m\pdv{V_m}{t} - \nabla \cdot (\Sigma_i \nabla \phi_i^2) + \chi_m I_{ion}(V_m,w) = I_i^{ext}    & \text{in } \Omega_{mus} \cross (0,T]
	\\
	\Sigma_i\nabla \phi_i^1 \cdot n = b_i   & \text{on } \partial \Omega_{mus} \cross (0,T]
	\\
	\Sigma_i\nabla \phi_i^2 \cdot n = b_i   & \text{on } \partial \Omega_{mus} \cross (0,T]
	\end{cases}
	\end{equation*}\\
	Subtracting:
	\begin{equation*}
	\begin{cases}
	- \nabla \cdot (\Sigma_i \nabla \varphi) = 0    & \text{in } \Omega_{mus} \cross (0,T]
	\\
	\Sigma_i\nabla \varphi \cdot n = 0   & \text{on } \partial \Omega_{mus} \cross (0,T]
	\end{cases}
	\end{equation*} \\
	That is a classical \emph{Laplace problem} with homogeneous Neumann boundary conditions. From theory \cite{salsa}, we know that the solution set is composed by all constant terms (remember that $\Sigma_i$ is positive definite). However, we must pay attention to the fact that $\varphi$ is a time-dependent function, even if time does not compare in the system. Thus, we can state:
	
	\begin{equation*}
	\exists \tilde{\varphi}:[0,T]\rightarrow \mathbb{R} \text{ such that } \varphi(x,t) = \tilde{\varphi}(t) \quad \forall x \in \Omega,\forall t \in [0,T] 
	\end{equation*}\\
	To conclude, we can observe now that if these two couples of solutions exist, then:
	\begin{equation*}
	\phi_i^1-\phi_i^2 = \phi_e^1-\phi_e^2=\tilde{\varphi} \quad \forall x \in \Omega, \forall t \in [0,T]
	\end{equation*} \qed
\end{proof}

\begin{remark}
For what concerns the regularity of $\varphi$, we can certainly state that, as a difference of two sufficiently regular functions, it belongs to the same class of regularity of the potentials if restricted to the sole time variable.
\end{remark}

\vspace{4mm}
\noindent We can conclude this analytical digression with an accomplished necessary and sufficient condition for the potentials solutions. \vspace{4mm}

\begin{corollary}
Suppose the couple ($\phi_i,\phi_e$) is a classical solution of the problem of definition \ref{def1} (for a certain $w$). The couple ($\tilde{\phi}_i,\tilde{\phi}_e$) of sufficiently regular real functions defined in $\Omega\cross[0,T]$ is another couple solution if and only if both $\tilde{\phi}_i,\tilde{\phi}_e$ differ respectively from $\phi_i,\phi_e$ for a time-dependent function $\varphi$ that belongs to the union of the regularity classes of the previous functions if restricted to time variable.
\end{corollary}

\begin{proof}
	The regularity statement is trivial and already discussed. The right implication is due to the previous theorem. Finally, the left implication follows what has been shown in equation \ref{phi_uniqueness}: it is enough to insert $\phi_i+\varphi$ and $\phi_e+\varphi$ in the Bidomain system to find out that $\varphi$ disappears and the remaining system is the same as the one with $\phi_i,\phi_e$, thus solved by hypothesis. \qed
\end{proof}
\vspace{4mm}
\subsection{Numerical correction}
Previous analytical results are crucial for what concerns the numerical computations since the Bidomain problem turns now to be \emph{not exactly} well-posed. Even if, in general, the right $V_m$ is most of the times achieved thanks to its uniqueness, our aim is to impose a further condition on the $\phi$ unknowns for the following two reasons:
\begin{enumerate}
	\item To pursue the achievement of the exact $\phi_i,\phi_e$ and not their values minus a constant. Useful for instance for the error analysis.
	\item To strengthen our Galerkin formulation that currently derives from a ill-posed problem and, for this reason, may show bad features as the generation of a ill-conditioned system or even a non-solvable system.
\end{enumerate}

\vspace{3mm}
\noindent We firstly observe that the additional condition should be applied only to one of the potentials, for instance to $\phi_i$. Indeed, the difference of the two possible solutions is $\varphi$ for both intracellular and extracellular potentials. Imposing $\varphi$ at each time-step implies the uniqueness imposition for both $\phi_i,\phi_e$. \\

\noindent The most common and simple strategies are the following:
\begin{enumerate}
	\item Imposition of the value of the function in a specific point.
	\begin{equation*}
	\phi_i(\bar{x},t) = \varphi(t) \quad \forall t \in [0,T]
	\end{equation*}
	\item Imposition of the function mean value.
	\begin{equation*}
	\int_\Omega \phi_i(x,t)\,dx = \varphi(t) \quad \forall t \in [0,T]
	\end{equation*}
\end{enumerate}

\noindent Notice that the first strategy would be useless if we keep working with an analytical and abstract weak formulation in Sobolev spaces. However, in the numerical context, we can assume certain regularities for the solution that let it makes sense and be the most common choice for numerical implementations. \\

\noindent As we will examine later on, it is demanding to implement the first strategy in a Dubiner context without losing the main system properties. Thus, we slightly change the first strategy and we instead opt for the imposition of a vector solution coefficient. For what concerns the FEM basis, this has the same meaning as before, provided that $\bar{x}$ is not a whatever point but a \emph{dof} point. On the other hand, for Dubiner basis, this has a completely different and abstract meaning: we remind that this time it has the role of modal coefficient. \\

\noindent Consider $\{u_j\}_{j=1 \dots N_h}$ as the list of the vector solution. As a consequence, we give the numerical version of the previous strategies:
\begin{enumerate}
	\item Imposition of a coefficient of the vector solution.
	\begin{equation*}
	u_l^k = \varphi(t^k) \quad \forall k \in \{1,K\}
	\end{equation*}
	\item Imposition of the function mean value.
	\begin{equation*}
	\sum_{j=1\dots N_h} u_j^k \, w_j = \varphi(t^k)\quad \forall k \in \{1,K\}
	\end{equation*}
\end{enumerate}	
where $l$ is a fixed value $\in \{1,N_h\}$ and $w_j$ stands for a suitable weight (that depends on the mesh geometry, the basis choice and the quadrature formula choice). \\

\noindent We remind that our aim is to impose such conditions \emph{directly} into the system. The easiest way would certainly be to impose these conditions \emph{after} the system resolution, as it has been reproduced in other past works. However, in this case, some issues related to the ill-posedness arise, especially ill-conditioning. Then, in the next sections we will illustrate how we managed to impose potential uniqueness only changing matrices and vectors coefficients before the resolution.

\subsubsection{Implementation of the first coefficient imposition} \label{first_coeff_implementation}
For simplicity, we choose $l=1$. Then, $u_l^k$ has the meaning of:
\begin{itemize}
	\item Value of $u$ in the first dof point, $\forall$ timestep $k$ (FEM)
	\item Fourier coefficient of $u$ w.r.t. the first Dubiner basis function, $\forall$ timestep $k$ (Dubiner)
\end{itemize}

\noindent What follows will be independent from basis choice. Suppose $c\in \mathbb{R}$ is the value to impose in the system $Au=\vec{b}$ for a certain timestep $k$. Since first coefficient occupies the first cell in the unknown vector and influences other coefficient values only through the first matrix column, we can switch the system from: \\

\begin{equation*}
A=\begin{bmatrix}
a_{11} & a_{12} & \dots & a_{1N} \\ 
a_{21} & a_{22} & \dots & a_{2N} \\ 
a_{31} &a_{32} & \dots & a_{3N} \\
\dots & \dots & \dots & \dots \\
a_{N1}  & a_{N2} & \dots & a_{NN}
\end{bmatrix} \quad \quad
b=\begin{bmatrix}
b_1 \\ b_2 \\ b_3 \\ \dots \\ b_N
\end{bmatrix}
\end{equation*}

to:

\begin{equation*}
\quad \quad  \quad \, \tilde{A}=\begin{bmatrix}
1 & 0 & \dots & 0 \\ 
0 & a_{22} & \dots & a_{2N} \\ 
0 &a_{32} & \dots & a_{3N} \\
\dots & \dots & \dots & \dots \\
0  & a_{N2} & \dots & a_{NN}
\end{bmatrix} \quad \quad
\tilde{b}=\begin{bmatrix}
c \\ b_2 -a_{21}c \\ b_3-a_{31}c \\ \dots \\ b_N-a_{N1}c
\end{bmatrix}
\end{equation*}
\vspace{4mm} \\
\noindent This is certainly correct since in the first system line $u_1=c$ is automatically imposed and, in the other lines, $u_1$ is no more treated as unknown but as a known data and then moved to the r.h.s. of the system. \\
The very advantage of this procedure is the conservation of $A$ symmetry. As we have previously anticipated, we discarded the nodal value strategy because, using Dubiner basis, we would have lost this crucial property. \\
Moreover, the value $c$ can be freely chosen, for instance from the exact solution (when error analysis needs to be executed) or a conventional fixed value as zero. \\ 

\noindent On the other hand, there are two main disadvantages. First of all, the system first line information has been deleted during this procedure. However, if the mesh is composed by many elements, this information is not essential and the solution behavior is practically the same as this information were provided. \\
Secondly, if the system is hugely ill-conditioned or even non-solvable (this is the case when we have homogeneous boundary conditions and/or homogeneous forcing terms), this imposition may have an overshooting effect that unbalances the solution. For these problems, a global imposition has to be adopted and that is the reason why we implemented the more complicated mean value imposition strategy.\\

\noindent The coefficient imposition procedure has been implemented in the script \texttt{assign\_phi\_i.m} that takes the $c$ value from the exact solution: \\

\begin{lstlisting}[language=Matlab,basicstyle=\small, numbers=left, numberstyle=\tiny,  name = assign_phi_i.m, frame=single]
function [A, b] = assign_phi_i (A, b, t, Data, femregion)

...
...
...

if (Data.fem(1)=='P')

   x = femregion.dof(1,1); % x-coordinate of the first dof point
   y = femregion.dof(1,2); % y-coordinate of the first dof point
   exact_coeff = eval(Data.exact_sol_i); % evaluation of exact sol


elseif (Data.fem(1)=='D')
   x0=femregion.dof(3,1); % bottom-left corner of the first element
   y0=femregion.dof(3,2);
   h=femregion.dof(1,1)-femregion.dof(3,1); % length of the element

   exact_coeff = 0;
   index = 1;
   
   ...
   ...
   ...

   % the first coeff is the L2 scalar product of uh with the first
   % basis function. To the get the right coeff, we compute the scalar 
   % product between the exact solution and the first basis function
   
   for k = 1:length(w_2D) % loop over 2D quadrature points
      %physical coords of the integration point
      x = x0 + h*node_2D(k,1);  
      y = y0 + h*node_2D(k,2);
      evalsol = eval(Data.exact_sol_i);
      exact_coeff = exact_coeff + evalsol*phi_dub(1,k,index).*w_2D(k);
   end

end


% we change the system coefficients to impose u(1)=exact_coeff
Nh = length(b);
b = b - A(:,1)*exact_coeff;
b(1) = exact_coeff;  
A(:,1) = zeros(Nh,1);
A(1,:) = zeros(1,Nh);
A(1,1) = 1;
\end{lstlisting}
\vspace{4mm}
\noindent To conclude this section, it is interesting to observe that the numerical imposition is done at \emph{every} time-step. This is confirmed from previous analytical theory as the difference $\varphi$ is a constant but depending on time: therefore, it is needed to fix it at every time-step.

\subsubsection{An analytical motivation for the mean value imposition method}
\noindent It is easy to realize that the procedure in section \ref{first_coeff_implementation} cannot be replicated for the mean unless losing symmetry. For instance, in the case of FEM basis and zero mean, a first line full of ones would imply also a first column full of ones and thus the resolution would be compromised. We should look for a different strategy. Let us start with a simple reference problem: \\
\begin{problem}[Reference zero-mean problem - strong form] If $\Omega$ open, bounded and sufficiently regular set, $f\in C^0(\bar{\Omega})$, find $u\in C^2(\Omega)\cap C^1(\bar{\Omega})$ such that:
\begin{equation*}
\begin{cases}
-\Delta{u}=f & \text{in } \Omega\\
\int_{\Omega} u = 0 & \text{in } \Omega \\
\nabla u \cdot n = 0 & \text{on } \partial \Omega \\
\end{cases}
\end{equation*}
\end{problem}\vspace{3mm}
\noindent For our scopes, it is convenient to move to the variational formulation:
\begin{problem}[Reference zero-mean problem - weak form]  \label{reference_problem} If $\Omega$ open and bounded set, $f\in L^2(\Omega)$, find $u\in H^1(\Omega)$ such that:
	\begin{equation*}
	\begin{cases}
	\mathlarger{\int}_{\Omega}\nabla u \cdot \nabla v = \mathlarger{\int}_{\Omega} fv \quad \forall v \in H^(\Omega)\\[2ex]
	\mathlarger{\int}_{\Omega}u = 0
	\end{cases}
	\end{equation*}
\end{problem}
\noindent As usual, elliptic regularity and $f,\Omega$ regularities implies that the weak solution is the classical solution as well. For this reason, let us focus only on the weak form. In addition, observe that if $\Omega$ is bounded, then $H^1(\Omega) \subset L^2(\Omega) \subset L^1(\Omega)$, therefore the second equation is justified. The next step is the study of the well-posedness.
\begin{lemma}
	The problem \ref{reference_problem} admits a weak solution $u$ if and only if the compatibility condition $\int_{\Omega} f = 0$ holds, in other words $f$ is a zero-mean function. Moreover, if $u$ exists, it is unique and it minimizes the Laplace energy functional $J(u) = \frac{1}{2}\int_{\Omega} |\nabla u |^2 - \int_{\Omega}fu$
\end{lemma} \vspace{1mm}
\begin{proof}
	Consider for the moment the first equation only. It is the Laplace problem with homogeneous Neumann boundary conditions. In a more general form, it is equivalent to a specific reaction-diffusion problem:
	\begin{equation*}
	\begin{cases}
	-\Delta{u} + \alpha u =f & \text{in } \Omega\\
	\nabla u \cdot n = 0 & \text{on } \partial \Omega \\
	\end{cases}
	\end{equation*}
	\begin{center}
		with $\alpha = 0$
	\end{center}
     We have already discussed that $\alpha=0$ is an eigenvalue of the Laplace operator with Neumann boundary conditions and its eigenspace is composed by all and only constant terms. From theorem 7.1.14 in \cite{gazzola}, we can state that, since $\alpha$ belongs to the spectrum and $f\in L^2(\Omega)$, existence of the weak solution holds if and only if the compatibility condition:
     \begin{equation*}
     \int_{\Omega}f = \int_{\partial \Omega} g = \int_{\partial_\Omega} 0 = 0 
     \end{equation*}
     holds. Then we solved the point about existence of the weak solution. \\
     
     \noindent For what concerns the uniqueness, we know from the same theorem that $u$ is unique except for other functions that differ from $u$ for an eigenfunction associated to $\alpha=0$. Since the eigenfunctions of zero are the functions that are constant \emph{a.e.}, we can state that the weak solution is unique minus a constant term. Then, if we add the second equation $\int_{\Omega}u = 0$, we achieve existence and uniqueness of the solution. \\
     
     \noindent Suppose now $u$ is the weak solution and $v$ another function $\in H^1(\Omega)$. Then:
     \begin{equation*}
     \begin{gathered}
     \exists w \in H^1(\Omega): \quad v = w+u \\
     \Rightarrow J(v)=J(u+w)= \frac{1}{2}\int_{\Omega}|\nabla u + \nabla w|^2 - \int_{\Omega}{fu}- \int_{\Omega}{fw} = \\
     = \overbrace{\frac{1}{2}\int_{\Omega}|\nabla u|^2 - \int_{\Omega}{fu}}^{J(u)} + \frac{1}{2}\int_{\Omega}|\nabla w|^2 + \overbrace{\int_{\Omega}\nabla u \cdot \nabla w - \int_{\Omega}{fw}}^{=0, \text { by def of weak solution}} =\\
     = J(u) + \frac{1}{2}\int_{\Omega}|\nabla w| ^2 \geq J(u)
     \end{gathered}
     \end{equation*}
     \qed
\end{proof}

\begin{remark}
	The minimization of the functional $J$ for the Laplace problem is a known fact. However, in this case where the sole Laplace-Neumann problem is not well-posed, this result was not trivial and thus it needed a check. Indeed, it is noteworthy to underline that minimization property holds but in a slightly different way: $u$ is not the absolute minimum point, every $u+k, k\in\mathbb{R}$ reaches the same minimum.
\end{remark}\vspace{4mm}

\noindent Previous well-posedness and minimization results implies that, if $u$ solves problem \ref{reference_problem}, then it is unique and, moreover, it is the unique zero mean value function that minimizes the functional $J(u)$. Thus, we can transform problem \ref{reference_problem} in another formulation:\\

\begin{problem}[Reference mean-value problem - 2] \label{reference_problem_2} Find $u\in H^1(\Omega)$ such that
	\begin{equation*}
	\begin{cases}
	J(u)=\displaystyle \min_{v\in H^1(\Omega)} J(v)\\
	I(u) = 0
	\end{cases}
	\end{equation*}
 where $f\in L^2(\Omega)$ and  \vspace{2mm}
	\begin{itemize}
		\item $J(u) = \frac{1}{2}\int_{\Omega} |\nabla u |^2 - \int_{\Omega}fu$
		\item $I(u) = \int_{\Omega} u $
	\end{itemize}
\end{problem}\vspace{3mm}

\noindent Surely a solution of problem \ref{reference_problem} is a solution of problem \ref{reference_problem_2}. The converse implication could be easily proved too. Then, the two problems are well-posed and completely equivalent. The advantage of the second form is that it consists in a minimization problem with constraints, a kind of problem that can be solved with generalized \emph{Lagrange Multipliers}. It means that:

\begin{equation*}
\exists \lambda \in \mathbb{R} \text{ such that} \quad <J'(u),v>+\lambda <I'(u),v>=0 \quad \forall v \in H^1(\Omega)
\end{equation*}\\
\noindent where $J',I'$ are the \emph{Frechét derivatives} of the two operators $J,I$ and $<\cdot,\cdot>$ represents the $H^1$ duality. Computing the derivatives, we indeed obtain:
\begin{equation*}
\exists \lambda \in \mathbb{R} \text{ such that} \quad \int_{\Omega}\nabla u \cdot \nabla v + \lambda \int_\Omega v = \int_{\Omega} fv \quad \forall v \in H^1(\Omega)
\end{equation*}\\

\noindent We can then formulate a third and last version of the reference problem:
\begin{problem}[Reference mean-value problem - 3]\label{reference_problem_3} Find $ u \in H^1(\Omega), \lambda \in \mathbb{R}$ such that:
	\begin{equation*}
	\begin{cases}
	\mathlarger{\int}_{\Omega}\nabla u \cdot \nabla v + \lambda \mathlarger{\int}_\Omega v = \mathlarger{\int}_{\Omega} fv \quad \forall v \in H^1(\Omega) \\[3ex]
	\mathlarger{\int}_{\Omega}u= 0
	\end{cases}
	\end{equation*}
\end{problem}

\noindent It may seem a very trivial result but actually it will be the very essence of our mean-value imposition strategy. First of all, let us check that existence and uniqueness properties have been conserved.  \newpage
\begin{lemma} \label{lemma_lagrange}
	Suppose previous assumptions on data are satisfied. Then there exists a couple solution $(u,\lambda)$ to the problem \ref{reference_problem_3}. Moreover, $\lambda=0$ and $u$ is unique. \\
\end{lemma}
\begin{proof}
	For what concerns existence, we can immediately realize that the solution $u$ of problem \ref{reference_problem} solves the problem \ref{reference_problem_3} with $\lambda=0$. Then the existence property holds because existence of problem \ref{reference_problem} has already been proved. \\
	
	\noindent Suppose now there exist two couples $(u_1,0), (u_2,\lambda)$ solutions of the problem and define $\varphi=u_2-u_1$. Then:
	
	\begin{equation*}
	\begin{cases}
	\int_{\Omega}\nabla u_1 \cdot \nabla v = \int_{\Omega} fv \quad \forall v \in H^1(\Omega) \\
	\int_{\Omega}\nabla u_2 \cdot \nabla v + \lambda \int_\Omega v = \int_{\Omega} fv \quad \forall v \in H^1(\Omega) \\
	\int_{\Omega}u_1 = \int_{\Omega}u_2 = 0
	\end{cases}
	\end{equation*}
		Subtracting:
	\begin{equation*}
	\begin{cases}
	\int_{\Omega}\nabla \varphi \cdot \nabla v + \lambda \int_\Omega v = 0 \quad \forall v \in H^1(\Omega) \\
	\int_{\Omega}\varphi = 0
	\end{cases}
	\end{equation*}
	
	\noindent If we assign $v=\varphi \in H^1(\Omega)$, then:
	\begin{equation*}
	\begin{cases}
	\int_{\Omega}|\nabla \varphi|^2 + \lambda \int_\Omega \varphi = \int_{\Omega}|\nabla \varphi|^2 = 0 \quad \forall v \in H^1(\Omega) \\
	\int_{\Omega}\varphi = 0
	\end{cases}
	\end{equation*}
	Then, $\norm{\nabla \varphi}_{L^2} = 0 $ implies $\varphi$ constant. But, since it has zero mean,  $\varphi = 0$ a.e.  \\
	To conclude, if $u_1=u_2$ a.e. as just proved, the previous system becomes:
	\begin{equation*}
	\lambda \int_{\Omega}v = 0 \quad \forall v \in H^1(\Omega)
	\end{equation*}
	that trivially implies $\lambda=0$	
\end{proof}
	\qed
	\vspace{3mm} \\
	\noindent This analytical digression was intended as a clarification of how problem \ref{reference_problem_3} can be considered as equivalent to the original problem \ref{reference_problem} (indeed, they are both well-posed and have the same solution). For this reason, the system modifications of the next section will be in some way justified by the previous results even if Bidomain problem is hugely more complicated than the simple Laplace problem.

\vspace{8mm}

\subsubsection{Implementation of the mean value imposition} \label{mean_value_implementation}
\noindent Following the problem \ref{reference_problem_3} new formulation, the basic idea is to consider $\lambda$ as a new coefficient of the vector solution, for instance the last one. The vector $u$ is now of dimension $N_h+1$. Let us define $d_i=\int_\Omega \varphi_i$ where $\varphi_i$ is the $i$-th basis function (both for FEM and Dubiner). Moreover, define $c$ as the imposed value for the mean. Then the discretized problem at a certain time-step turns to be:

\begin{problem}[Discretized mean-value imposition problem]
	Find $\{u_i\}_{i=1\dots N_h+1}$ such that:
	\begin{equation*}
	\begin{cases}
	\mathlarger{\sum}_{i=1}^{N_h} u_i \mathlarger{\int}_{\Omega}\nabla\varphi_i \cdot \nabla \varphi_j + \lambda \,d_j=\mathlarger{\int}_{\Omega}f\phi_j \quad \forall j=1\dots N_h \\[3ex]
	\mathlarger{\sum}_{i=1}^{N_h} u_i \,d_i = c
	\end{cases}
	\end{equation*}
\end{problem}
\vspace{3mm}
\noindent Reminding that $\lambda=u_{N_h+1}$, the previous problem consists in the system transformation from: \\
\begin{equation*}
A=\begin{bmatrix}
a_{11} & a_{12} & \dots & a_{1N} \\ 
a_{21} & a_{22} & \dots & a_{2N} \\ 
\dots & \dots & \dots & \dots \\
a_{N1}  & a_{N2} & \dots & a_{NN}
\end{bmatrix} \quad \quad
b=\begin{bmatrix}
b_1 \\ b_2 \\ \dots \\ b_N
\end{bmatrix}
\end{equation*}

to:

\begin{equation*}
\quad \quad  \quad \, \tilde{A}=\begin{bmatrix}
a_{11} & a_{12} & \dots & a_{1N} & d_1\\ 
a_{21} & a_{22} & \dots & a_{2N} & d_2 \\ 
\dots & \dots & \dots & \dots & \dots \\
a_{N1}  & a_{N2} & \dots & a_{NN} & d_N \\
d_1 & d_2 & \dots & d_N & 0
\end{bmatrix} \quad \quad
\tilde{b}=\begin{bmatrix}
b_1 \\ b_2 \\ \dots \\ b_N \\ c
\end{bmatrix}
\end{equation*}
\vspace{3mm} \\
\noindent First of all, observe that symmetry is conserved. Moreover, this time no line has been deleted, all the information is conserved. To conclude, we remind from lemma \ref{lemma_lagrange} that $\lambda$ is a fake unknown, its value is always zero and, if it is not, there is some issue. \\

\noindent The implementation of such transformation is located in the \texttt{assign\_null\_average.m} script. Some lines follow: \\

\begin{lstlisting}[language=Matlab,basicstyle=\small, numbers=left, numberstyle=\tiny,  name = assign_null_average.m, frame=single]
function [A, b] = assign_null_average (A, b, Data, femregion)

Nh = length(b)/2;

if (Data.fem(1)=='P')

   ...
   ...
   ...

   for k = 1:length(w_2D)
      coeff = coeff + phi_fem(1,k,1).*w_2D(k);
   end

   for i = 1:Nh
      A(i,2*Nh+1)=coeff;
      A(2*Nh+1,i)=coeff;
   end


elseif (Data.fem(1)=='D')

   ...
   ...
   ...

   for p = 1:femregion.nln
      p_int = 0;
      for k = 1:length(w_2D)
         p_int = p_int + phi_dub(1,k,p).*w_2D(k);
      end
      coeff(p)=p_int;
   end

   for i = 1:femregion.nln:Nh
      A(2*Nh+1,i:i+femregion.nln-1)=coeff';
      A(i:i+femregion.nln-1,2*Nh+1)=coeff;
   end

end

b = [b;0];

\end{lstlisting}
\vspace{8mm}
\noindent Some comments:
\begin{itemize}
	\item \texttt{Nh = length(b)/2} at line 3 because the original system is a block matrix system. We remind that this transformation concerns only $\phi_i$, then it applies to the first half of the system only.
	\item $c$ is chosen to be zero (line 42), its value does not come from an exact solution because the mean-value strategy has been adopted only for realistic simulation with no exact solutions (see \ref{uniqueness_results}). 
	\item We avoided to compute all $d_i$ for FEM basis as they all have the same value.
	\item On the other hand, for Dubiner basis, $d_i$ values are different. However, it is not needed to compute these values for all the global polynomials as they repeat for every element. For this reason, we only iterate over the local degrees of freedom.
\end{itemize}
\vspace{6mm}

\subsubsection{Results} \label{uniqueness_results}
\noindent As already discussed, the mean-value imposition was implemented and adopted only for very ill-conditioned systems. For all other cases, the coefficient imposition worked perfectly. This is why, in our research, we chose to adopt:
\begin{itemize}
	\item the coefficient imposition for error analysis simulations (as boundary conditions and forcing terms were never homogeneous)
	\item the mean value imposition for realistic simulations (as boundary conditions and forcing terms were essentially homogeneous)
\end{itemize}\vspace{2mm}

\noindent In conclusion, we successfully imposed the potentials uniqueness directly into the system. For what concerns our two initial objectives, we achieved both them since:
\begin{itemize}
	\item Error analysis studies demonstrated that $\phi_i,\phi_e$ potentials converge to the exact potentials as $V_m$ and $w$ do.
	\item Conditional numbers turn to be considerably decreased. Our measures proved that they passed from $\approx 10^{17}$ to $\approx 10^{7}$ for both the strategies.
\end{itemize} 

    \newpage
    \printbibliography

\end{document}

    
	
    
    
    
    
    
    
    
    
    
    
    
    \newpage
    \printbibliography

\end{document}
