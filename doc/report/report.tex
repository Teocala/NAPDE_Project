\documentclass[a4paper,11pt]{article}
\usepackage[a4paper,top=5cm,bottom=4cm,left=3.6cm,right=3.6cm]{geometry}
\usepackage[T1]{fontenc}
\usepackage[utf8]{inputenc}
\usepackage[english]{babel}
\usepackage{graphicx}
\usepackage{amsmath}
\usepackage{amsfonts}
\usepackage{amsthm}
\usepackage{enumerate}
\usepackage{enumitem}
\usepackage{physics}
\usepackage{bm}
\usepackage{setspace}
\usepackage{caption}
\usepackage{mathtools}
\usepackage{subcaption}
\captionsetup{compatibility=false}

\usepackage[
backend=biber,
style=numeric,
citestyle=numeric,
]{biblatex}

\addbibresource{./report_bib.bib}
\usepackage[autostyle]{csquotes}

\captionsetup{font=footnotesize}


\newtheorem{definition}{Definition}
\newtheorem{prop}{Proposition} 



\begin{document}
	
	%% FRONT PAGE
	\begin{titlepage}
	    \thispagestyle{empty}
	    \newgeometry{left=2cm,right=2cm}
	    \begin{center}
	    	\includegraphics[width = 4cm]{./polimi-logo.png}\\ \vspace{3mm}
	    	\normalsize{\textsc{Course of Numerical Analysis for Partial Differential Equations}}
	    	
	    	\vspace{20mm}
	    	\rule{15cm}{0.1mm} \\ \vspace{4.5mm}
	    	 \Huge{\textbf{HIGH-ORDER DISCONTINUOS GALERKIN METHOD FOR THE BIDOMAIN PROBLEM OF CARDIAC ELECTROPHYSIOLOGY}} \\
	    	\rule{15cm}{0.1mm}
	    \end{center}
	    	\vspace{25mm}
	    	
	    	\Large{
	    	\hspace{11mm} \emph{Authors:} \hspace{5mm} \textsc{ \quad Federica Botta, Matteo Calafà}} \newline
	    	 
	    	\large{\vspace{3mm} 
	    	 \qquad \emph{Supervisors:} \hspace{12mm} \textsc{ Christian Vergara, Paola Antonietti}}
	          \\
	    	\vspace{20mm}
	    \begin{center}
	    	\large{\textsc{A.Y. 2020/2021}}
	    \end{center}
	\end{titlepage}



    \restoregeometry
    
    %% TABLE OF CONTENTS
    \tableofcontents
    \newpage
    
    %% ABSTRACT
    \section{Introduction}
    \subsection{Abstract}
    The aim of the project is to study and implement a suitable numerical scheme for the resolution of the \emph{Bidomain Problem}, a famous system of equations that has been developed in the context of the electrophysiology of human heart. \\
    This work is basically the continuation of a two-years-long study carried out by three past course projects (\cite{bagnara}, \cite{andreotti}, \cite{marta}). In particular, the very goal of this project is to improve the results obtained in \parencite{marta} (\citeauthor{marta}) for the Bidomain model. In fact, even if a \emph{Discontinuous Galerkin} discretization has been successfully implemented, results are not satisfactory from the point of view of stability and convergence. We think this notice is noteworthy as this work is primarily based on these provided data and codes. Through this article, it will be illustrated how we managed to solve these problems extending, optimizing and correcting these past numerical strategies.
    
    %% PHYSICAL PROBLEM
    \subsection{The physical problem}
    We intend to present the physical meaning of the Bidomain equations only briefly since it has already been widely shown in the previous project (\citeauthor{marta}). For a more complete explanation, we instead refer to \cite{acta}.\\
    The mechanical contraction and expansion of human heart has its origin in the \emph{electrical activation} of the cardiac cells. At every heart-beat, myocyties are activated and deactivated following a characteristic electrical cycle (fig \ref{potential_cycle}). 
    
    
    \begin{figure}[h]
    \begin{center}
    \includegraphics[width = 7cm]{./potential_cycle.png}
    \caption{Membrane potential in function of time (one cardiac cycle)}
    \label{potential_cycle}
    \end{center}
    \end{figure}
    
    \noindent The cell is initially at rest ($-90mV$, step 4). At a certain point, its potential increases rapidly ($\approx2ms$) and reaches the value of $+20mV$: the cell is activated. Later, a plateau near $0mV$ is observed and then a slow repolarization to the initial potential. \\
    From a microscopical point of view, we could study the dynamics acting in each single cell (as a consequence of the passage of chemical ions through specific channels, e.g. $Ca2+,Na+,K+$). From a macroscopical point of view, instead, one can observe it as a continuous electrical diffusion over the entire cardiac surface. Even if this consists in a very rapid phenomenon, the study of such propagation could be very interesting in order, for instance, to detect diseases in sick patients.
    
    \subsection{Mathematical models}
    Starting from the circuit in figure \ref{electrical_circuit}, applying some general electromagnetism laws and some calculations, the Bidomain model has been formulated (see \parencite{acta} for more details and/or \parencite{colli_franzone} for the complete passages).
    
    \begin{figure}[h]
    	\begin{center}
    		\includegraphics[width = 7cm]{./electrical_circuit.png}
    		\caption{Simplified circuit to model the intracellular and extracellular potentials dynamics}
    		\label{electrical_circuit}
    	\end{center}
    \end{figure}
    
    \noindent The general formulation is then: \vspace{3mm}
    \begin{definition}[Bidomain model]
	\begin{equation*}
	\begin{cases}
	\chi_m C_m\pdv{V_m}{t} - \nabla \cdot (\Sigma_i \nabla \phi_i) + \chi_m I_{ion} = I_i^{ext}    & \text{in } \Omega_{mus} \cross (0,T]
	\\
	-\chi_m C_m\pdv{V_m}{t} - \nabla \cdot (\Sigma_e \nabla \phi_e) - \chi_m I_{ion} = -I_e^{ext}    & \text{in } \Omega_{mus} \cross (0,T]
	\end{cases}
	\end{equation*}
    \end{definition}
	\vspace{3mm}
	
	where:
	\begin{itemize}[label=\textendash]
		\item $\bm{\phi_i, \phi_e}$ are the \emph{Intracellular and Extracelllular Potentials} (unknowns)
		\item $V_m = \phi_i-\phi_e$ is the \emph{Trans-membrane Potential}
		\item $\chi_m,C_m$ are known constants and $\Sigma_i, \Sigma_e$ are known constant tensors 
		\item $I_i^{ext},I_e^{ext}$ are applied currents
		\item $I_{ion}$ is the \emph{Ionic Current}
		\item $\Omega_{mus}$ is the cardiac domain (myocardium + endocardium + epicardium)
	\end{itemize}
    
    \vspace{4mm}
    \noindent Actually, this system is not complete since it misses boundary and initial conditions and a suitable model for $I_{ion}$. Initial conditions and Neumann boundary conditions for $\phi_i$ and $\phi_e$ are then imposed. For the definition of $I_{ion}$, instead, a \emph{reduced ionic model} is chosen, in particular the \emph{FitzHugh-Nagumo model}. Summing up:
    
    \begin{definition}[Bidomain + FitzHugh-Nagumo model with Neumann boundary conditions]\label{def1}
    	\begin{equation*}
    	\begin{cases}
    	\chi_m C_m\pdv{V_m}{t} - \nabla \cdot (\Sigma_i \nabla \phi_i) + \chi_m I_{ion}(V_m,w) = I_i^{ext}    & \text{in } \Omega_{mus} \cross (0,T]
    	\\
    	-\chi_m C_m\pdv{V_m}{t} - \nabla \cdot (\Sigma_e \nabla \phi_e) - \chi_m I_{ion}(V_m,w) = -I_e^{ext}    & \text{in } \Omega_{mus} \cross (0,T]
    	\\
    	I_{ion}(V_m,w)=kV_m(V_m-a)(V_m-1)+w & \text{in } \Omega_{mus} \cross (0,T]
    	\\
    	\pdv{w}{t} = \epsilon(V_m-\gamma w)  & \text{in } \Omega_{mus} \cross (0,T]
    	\\
    	\Sigma_i\nabla \phi_i \cdot n = b_i   & \text{on } \partial \Omega_{mus} \cross (0,T]
    	\\
    	\Sigma_e\nabla \phi_e \cdot n = b_e   & \text{on } \partial \Omega_{mus} \cross (0,T]
    	\\
    	\text{Initial conditions for } \phi_i,\phi_e, w & \text{in } \Omega_{mus}\cross\{t=0\}
    	\end{cases}
    	\end{equation*}
    \end{definition}
    \vspace{3mm}
    where:
    \begin{itemize}[label=\textendash]
    	\item $\bm{w}$ is the \emph{gating variable} (unknown)
    	\item $k,a,\epsilon,\gamma$ are known constants
    	\item $b_i,b_e$ are the boundary conditions data
    	\item $n$ is the outward normal vector
    \end{itemize}

    \vspace{4mm}
    \noindent From now on, the system of definition \ref{def1} will be the reference analytical problem for the development of numerical schemes.\\
    To conclude, there exist other famous and useful models, such as the \emph{Monodomain model}. But this is just a simplification of the Bidomain as in this case it is assumed that $\phi_i$ and $\phi_e$ are proportional. However, thanks to its simplicity, we often tested the code starting from the Monodomain implementation of the project \cite{andreotti} instead of analyzing directly the Bidomain.

    \subsection{A short discussion about the past works issue}
    As we have already introduced, our project initially aimed to continue and improve the work of a previous project (\parencite{marta}). \\
    Results obtained using unitary parameters, namely $\chi_m =\Sigma_i= \Sigma_e= C_m= k = \epsilon= \gamma= a=1$, were actually quite satisfactory. Instead, the choice of more realistic/experimental values for the parameters (that are often very big or very small) caused bad consequences to the accuracy of the schemes or even to their stability. In particular, we observed that the choice of $C_m \approx 10^{-2}$ highly compromised the stability of the numerical schemes.
    This issue heavily limits the use of the code for research and/or experimental simulations as it guarantees convergence to the right solution only in few and non-realistic problems.  \\

    \noindent After a while, we realized that an inverted sign of the FitzHugh-Nagumo model formula occurred in \cite{acta}. \\

    \noindent This oversight was not only essential for the fidelity to the real phenomena but also crucial for the well-posedness of the problem.\\ 
    We could give two motivations to reinforce this last statement: first of all, if we consider the well known study of \citeauthor{bourgault} (\cite{bourgault}), the conditions required for the well-posedness of the Bidomain problem are not satisfied if the sign is inverted (neither for the existence, hypothesis H4, nor for the uniqueness).\\ Secondly, suppose to discretize the Bidomain problem in time and to treat the non-linear term semi-implicitly, as it will be done in the following sections. Then, if we fix the timestep and if data from the previous timestep are given, we achieve a linear problem that can be easily switched into a weak formulation. From this analysis, we could observe that if $V_m<a$ or $V_m>1$ and $C_m$ is sufficiently small, the associated bilinear form is not coercive. \\

    \noindent This second motivation, even if not very formal and with a mixed approach, is particularly interesting since a few confirmations occurred during simulations, as shown for instance in figure \ref{sol_ill_posed}.


\begin{figure}[h]
	\begin{center}
		\includegraphics[width = 11cm]{./sol_ill_posed.jpg}
		\caption{Comparison between exact and computed solution with inverted FitzHugh-Nagumo model: huge errors arise only when $Vm<0$}
		\label{sol_ill_posed}
	\end{center}
\end{figure}
    
    \noindent In conclusion, the issue of the past works \cite{andreotti}, \cite{marta} had not a numerical nature as expected, but instead an analytical origin due to an ill-posed problem.



    \section{Semi discretized numerical methods}
    \subsection{DG discrete formulation}
    We have seen the Bidomain model in a complete form in definition \ref{def1}. We now introduce a triangulation $\tau_h$ over $\Omega$, with $\mathcal{F} _h=\mathcal{F} _h^I \cup \mathcal{F} _h^B$ set of the faces of the elements, which includes the internal and boundary faces respectively, and the DG space $V_h^k = \{v_h \in L^2(\Omega) : v_h|_\mathcal{K} \in \mathbb{P}^{k}(\mathcal{K})  \quad \forall \mathcal{K} \in \tau_h \}$, where k is the degree of the piecewise continuous polynomial. We obtain the semi discrete DG formulation:\vspace{4mm}
    \begin{equation*}
    \text{For any time } t \in [0,T] \text{ find } \Phi_h(t)=[\phi_i^h(t),\phi_e^h(t)]^T \in [V_h^k]^2 \text{ and } w_h(t) \in V_h^k \text{ such that}\vspace{5mm} \\
    \end{equation*}
    \begin{enumerate}
    \item 
    \begin{equation*}
    \begin{gathered}
    \sum_{K \in \tau_h} \int_K{ \chi_m C_m \pdv{V_m^h}{t} V_h dw}+a_i(\phi_i^h,v_h)+\sum_{K \in \tau_h} \int_K{ \chi_m k (V_m^h-1)(V_m^h-a) V_m^h v_h dw}+\\
    +\sum_{K \in \tau_h} \int_K{ \chi_m w_h v_h dw}=(I_i^{ext},v_h) \qquad \forall v_h \in V_h^p\\
    \end{gathered}
    \end{equation*}
    \item
    \begin{equation*}
    \begin{gathered}
    -\sum_{K \in \tau_h} \int_K{ \chi_m C_m \pdv{V_m^h}{t} V_h dw}+a_e(\phi_e^h,v_h)-\sum_{K \in \tau_h} \int_K{ \chi_m k (V_m^h-1)(V_m^h-a) V_m^h v_h dw}+\\
    -\sum_{K \in \tau_h} \int_K{ \chi_m w_h v_h dw}=(-I_e^{ext},v_h) \qquad \forall v_h \in V_h^p\\
    \end{gathered}
    \end{equation*}
    \item
    \begin{equation*}
    \sum_{K \in \tau_h} \int_K{\pdv{w_h}{t}v_h dw}=\sum_{k \in \tau_h} \int_K{\epsilon (V_m^h-\gamma w_h) v_h dw} \qquad \forall v_h \in V_h^p\\
    \end{equation*}
    \end{enumerate}
    \vspace{5mm}
    where:
    \vspace{3mm}
    \begin{equation*}
    \begin{aligned}
    \bullet& \quad a_k(\phi_k^h,v_h)=\sum_{K \in \tau_h} \int_K{(\Sigma_k \nabla_h \phi_k^h) \cdot \nabla_h v_h dw}-\sum_{F \in \mathcal{F}_h^I} \int_F { \{\{\Sigma_k \nabla_h \phi_k^h \}\} \cdot [[v_h]] d\sigma}+\\
    &-\delta \sum_{F \in \mathcal{F}_h^I} \int_F{ \{\{\Sigma_k \nabla_h v_h\}\} \cdot [[\phi_k^h]]d\sigma}+\sum_{F \in \mathcal{F}_h^I}\int_F {\gamma [[\phi_k^h]] \cdot [[v_h]] d\sigma} \qquad k=i,e\\
    \newline
    \bullet& \quad (I_i^{ext},v_h)=\sum_{K \in \tau_h} \int_K {I_i^{ext} v_h dw}+\int_{\partial w}{b v_h d\sigma}\\
    \newline
    \bullet& \quad (-I_e^{ext},v_h)=-\sum_{K \in \tau_h} \int_K {I_e^{ext} v_h dw}+\int_{\partial w}{b v_h d\sigma}
    \end{aligned}
    \end{equation*}
    
    \noindent Moreover, according to the choice of the coefficient $\delta$, we can define:
    \begin{itemize}
    \item $\delta=1$: Symmetric Interior Penalty method (SIP)
    \item $\delta=0$: Incomplete Interior Penalty method (IIP)
    \item $\delta=-1$: Non Symmetric Interior Penalty method (NIP) 
    \end{itemize}
     \vspace{2mm}
    \noindent And $\gamma := \alpha \frac{k^2}{h}$ (\emph{"Stabilization parameter"}),$ \quad \alpha \in \mathbb{R}$ to be chosen high enough. \\
    
    \noindent To analyze the formulation more in details see \cite{marta}.
    \vspace{4mm}
    \subsection{Algebraic formulation}
    Taking $\{\varphi_j\}_{j=1}^{N_h}$ base of $V_h^k$, so that we can write
    \begin{equation*}
    \begin{gathered}
    \Phi_h(t) = \begin{bmatrix} \phi_i^h(t) \\ \phi_e^h(t) \end{bmatrix} = \begin {bmatrix}\sum_{j=1}^{N_h} \phi_{i,j}(t)\varphi_j \\ \sum_{j=1}^{N_h} \phi_{e,j}(t)\varphi_j \end{bmatrix}\\
    w_h(t) = \sum_{j=1}^{N_h}w_j(t)\varphi_j\\
    V_m^h(t)=\sum_{j=1}^{N_h} V_{m,j}(t) \phi_j=\sum_{j=1}^{N_h}(\phi_{i,j}(t)-\phi_{e,j}(t))\varphi_j
 \end{gathered}
 \end{equation*}
 Then, we introduce the matrices:
 \begin{equation}\label{matrices}
\begin{rcases}
(V_k)_{ij} &= \int_{w}\nabla\varphi_j \cdot \Sigma_k \nabla \varphi_i 
\\ (I_k^T)_{ij} &= \sum_{F \in F_h^I} \int_{F} \{\{\Sigma_k \nabla\varphi_j\}\} \cdot [[\varphi_i]] 
\\ (I_k)_{i,j} &= \sum_{F \in F_h^I} \int_{F} [[\varphi_j]] \cdot \{\{\Sigma_k \nabla \varphi_i\}\}
\\(S_k)_{i,j} &= \sum_{F \in F_h^I} \int_{F} \gamma_k[[\varphi_j]] \cdot [[\varphi_i]]
\end{rcases}
\begin{gathered}
\quad A_k = (V_k -I_k^T - \theta I_k + S_k)\\
k=i,e \\
\end{gathered}
\end{equation}
\begin{equation*}
\gamma_k\vert_F = (n_F^T \, \Sigma_k \, n_F) \,\gamma, \quad n_F \text{ outward normal vector of F}
\end{equation*}

\begin{equation}
A_i \qquad{\text{Intra-cellular stiffness matrix}}
\end{equation}
\begin{equation}
A_e \qquad{\text{Extra-cellular stiffness matrix}}
\end{equation}
\begin{equation}
M_{ij} = \sum_{K \in \tau_h}\int_K\varphi_j\varphi_i \qquad{\text{Mass matrix}}
\end{equation}
\begin{equation}
C(u_h)_{ij} =  \sum_{K \in \tau_h} \int_K \chi_m k(u_h-1)(u_h-a)\varphi_j\varphi_i \qquad{\text{Non-linear matrix}}
\end{equation}
\begin{equation}
F_k=\begin{bmatrix} F_{i,k} \\ F_{e,k} \end{bmatrix}=\begin{bmatrix} \int_{w} I_i^{ext}\varphi_k - \sum_{F \in F_h^B} \int_F b_i\varphi_k \\ - \int_{w} I_e^{ext}\varphi_k - \sum_{F \in F_h^B} \int_F b_e\varphi_k \end{bmatrix}
\end{equation}
\vspace{3mm} \\
Therefore, our semi-discrete algebraic formulation is: \vspace{3mm} \newline
find $\Phi_h(t)=[\phi_i^h(t),\phi_e^h(t)]^T \in [V_h^k]^2$ and $w_h(t) \in V_h^k$ for any $t \in (0;T]$ such that:
\begin{equation}\label{algebraic}
\begin{gathered}
\chi_m Cm M \dot{V_m^h}+A_i \phi_i^h+C(V_m^h) V_m^h+\chi_m M w_h=F_i^h\\
-\chi_m Cm M \dot{V_m^h}+A_e \phi_e^h-C(V_m^h) V_m^h-\chi_m M w_h=F_e^h\\
M \dot{w}_h(t)=\epsilon M (V_m^h(t)-\gamma w_h(t))
\end{gathered}
\end{equation}
 \vspace{5mm} \\
 Rewriting it with block matrices: \vspace{5mm} \newline
 find $\Phi_h(t)=[\phi_i^h(t),\phi_e^h(t)]^T \in [V_h^k]^2$ and $w_h(t) \in V_h^k$ for any $t \in (0;T]$ such that:
 \begin{equation}\label{block_matrix}
 \begin{gathered}
 \chi_mC_m \begin{bmatrix}M &-M \\ -M & M \end{bmatrix}
	\begin{bmatrix}\bm{\dot{\phi}_i^h(t)} \\ \bm{\dot{\phi}_e^h(t)} \end{bmatrix}
	 + \begin{bmatrix}A_i & 0 \\ 0 & A_e \end{bmatrix}
	 \begin{bmatrix}\bm{\phi_i^h(t)} \\ \bm{\phi_e^h(t)} \end{bmatrix} +\\
	   \begin{bmatrix}C(V_m^h) & -C(V_m^h) \\ -C(V_m^h) & C(V_m^h) \end{bmatrix} 
	   \begin{bmatrix} \bm{\phi_i^h(t)} \\ \bm{\phi_e^h(t)}  \end{bmatrix} 
	   +\chi_m \begin{bmatrix}M & 0 \\ 0 & -M \end{bmatrix} 
	   	\begin{bmatrix}w_h(t) \\ w_h(t) \end{bmatrix} = 
	   	\begin{bmatrix} F_i^h \\ F_e^h\end{bmatrix}\\
	   \dot{w}_h(t)=\epsilon (V_m^h(t)-\gamma w_h(t))
\end{gathered}
\end{equation}
\vspace{5mm}
\section{Dubiner Basis}
    So far, we have described a general semi-discrete discontinuous formulation without examining which basis to use to generate the $V_h^k$ space. Usually, the common choice consists in the classical hat functions from FEM, even if they need to be modified in order to be used in a discontinuous context. It is also one of the simplest choices, for this reason our provided code was initially implemented with this basis. However, the very novelty of this study is the adoption of a new kind of basis, completely different from the previous and commonly known as "\emph{Dubiner Basis}" \cite{dubiner}. \\
    How we will soon see, the peculiarity of this family of functions is that it consists of orthogonal polynomials defined on the reference triangle
    \begin{equation}
    \hat{K}=\{ (\xi, \eta) : \xi, \eta \ge 0,	\xi+\eta \le 1 \}
    \end{equation}
    and not on the reference square
    \begin{equation}
    \quad \quad \hat{Q}=\{ (a, b) : -1 \le a \le 1, -1 \le b \le 1 \}
    \end{equation}
    Formally, if we consider the transformation from $\hat{Q}$ to $\hat{K}$
    \begin{equation}\label{transformation_formula}
    \xi:=\frac{(1+a)(1-b)}{4},  \eta:=\frac{(1+b)}{2}
    \end{equation}
    
    \begin{figure}[h]
    \begin{center}
    \includegraphics[width = 7cm]{./transformation.png}
    	\caption{Transformation between the reference square to the reference triangle}
    	\label{transformation}
    \end{center}
    \end{figure}
    
    \noindent the Dubiner basis is the transformation of a suitable basis initially defined on the reference square. This initial basis is simply obtained with a two dimensional modified tensor product of the Jacobi polynomials on the interval $(-1,1)$.
    \begin{definition}[Jacobi polynomials]
    The Jacobi polynomials of coefficients $\alpha,\beta \in \mathbb{R}$ evaluated in $z\in (-1,1)$ are:
    \begin{itemize}[label=\textendash]
    \item $n=0$
    \begin{equation}
    J_0^{\alpha,\beta}(z)=1
    \end{equation}
    \item $n=1$
    \begin{equation}
    J_1^{\alpha,\beta}(z)=\frac{1}{2}(\alpha-\beta+(\alpha+\beta+2)\cdot z);
    \end{equation}
    \item $n\ge2$
    \newline
    \begin{equation}
    \begin{gathered}
    \begin{aligned}
    J_n^{\alpha,\beta}(z)=\sum_{k=2}^{n} \Big[&\frac{(2k+\alpha+\beta-1)(\alpha^{2}-\beta^{2})}{2k(k+\alpha+\beta)(2k+\alpha+\beta-2)}+ \\ &\frac{(2k+\alpha+\beta-2)(2k+\alpha+\beta-1)(2k+\alpha \beta)}{2k(k+\alpha+\beta)(2k+\alpha+\beta-2)} J_{k-1}^{\alpha,\beta}(z) +
    \\-&\frac{2(k+\alpha-1)(k+\beta-1)(2k+\alpha+\beta)}{2k(k+\alpha+\beta)(2k+\alpha+\beta-2)} J_{k-2}^{\alpha,\beta}(z) \Big]
    \end{aligned}
    \end{gathered}
    \end{equation}
    \end{itemize}
    \end{definition}
    \vspace{5mm}
    \noindent An important property of these polynomials is:
    \begin{prop}\label{jac}
    $J_i^{\alpha,\beta}(\cdot)$ is orthogonal w.r.t. the Jacobi weight $w(x)=(1-x)^\alpha(1+x)^\beta$:
    \begin{equation}
    \int_{-1}^{1}{(1-x)^\alpha(1+x)^\beta J_m^{\alpha,\beta} J_q^{\alpha,\beta}(x)dx}=\frac{2}{2m+1} \delta_{mq} 
    \end{equation}
    \end{prop}
    
    \noindent Thanks to this definition, we can now define explicitly the Dubiner basis.
    \begin{definition}[Dubiner Basis] \label{dubiner}
    The Dubiner basis that generates the space $\mathbb{P}^p(\hat{K})$ of the polynomials of degree $p$ over the reference triangle is the set of functions:
    \begin{equation}
    \begin{split}
    & \quad \quad \quad \phi_{ij}: \hat{K} \rightarrow \mathbb{R} \\
    \phi_{ij}(\xi,\eta) :&= c_{ij}(1-b)^j J_i^{0,0}(a) J_j^{2i+1,0}(b)=
    \\&=c_{ij} 2^j (1-\eta)^j J_i^{0,0}(\frac{2\xi}{1-\eta}-1) J_j^{2i+1,0} (2\eta-1)
    \end{split}
    \end{equation}
    for $i,j=1,\dots,p$ and $i+j \le p$, where
    \begin{equation}
    c_{ij} := \sqrt{\frac{2(2i+1)(i+j+1)}{4^i}}
    \end{equation}
    and $J_i^{\alpha,\beta}(\cdot)$ is the i-th Jacobi polynomial
    \end{definition}
    
    \vspace{4mm}
    \noindent As we have anticipated
    \begin{prop}
    The Dubiner basis is orthonormal in $L^2(\hat{K})$ $\forall p$:
    \begin{equation}
    \int_{\hat{K}}{\phi_{ij}(\xi,\eta)\phi_{mq}(\xi,\eta) d\xi d\eta}=\delta_{im}\delta_{jq}
    \end{equation}
    \end{prop}
    \vspace{4mm}
    \noindent As a consequence, after we successfully implemented the code with Dubiner basis and computed the matrixes, we obtained a diagonal mass matrix (figure \ref{mass})
    
    \begin{figure}[ht]
    \begin{center}
    \includegraphics[width = 7cm]{./mass_dubiner.jpg}
    	\caption{Non-zero elements in the mass matrix when adopting Dubiner basis}
    	\label{mass}
    \end{center}
    \end{figure}

    \noindent It is noteworthy to point out that transformation \ref{transformation_formula} is bijective, it can be inverted but it needs some care. The natural inverse would be:
    
    \begin{equation}
    a = \frac{2\xi}{1-\eta}-1 \quad \quad b = 2\eta-1
    \end{equation}
    \vspace{2mm} \\
    \noindent that has already been used for the definition \ref{dubiner}. However, it is not defined for $\eta=1$, that means for the sole point $(0,1)$ of the reference triangle. To solve this issue, it is enough to prolong the function with continuity to this special point. For the code implementation, it is suggested avoiding evaluations in the exact point or adding an \emph{if} condition. We opted for the second solution. \\
    
    \noindent In general, the orthogonality property implies some good numerical properties, not only the diagonalization of the mass matrix. For instance, in \cite{antonietti} interesting bounds for the conditional number can be viewed. For this reason, we opted for this choice aiming to improve the previous results, at least from the space discretization side. \\
    However, there are also some difficulties arising when one chooses to abandon the familiar FEM basis. First of all, the coefficients of a discretized function has only \emph{modal} meaning and they no more represent the \emph{nodal} values of the function itself. This fact needs some extra work when one needs to switch from the continuous functions to the discretized functions and viceversa, as it will be shown in the paragraph \ref{subsection_implementation}. Secondly, one can notice that these functions are not boundary conditions friendly. What we mean is that, if compared to FEM basis, they have no particular properties on the boundary to let easily impose homogeneous boundary conditions. Thus, they should be again transformed, this time in a \emph{boundary adapted} form. We refer to \cite{napde} for a short description of this procedure. Fortunately, we do not need to set this transformation as in the discontinuous formulation boundary conditions (both Dirichlet and Neumann) are imposed only weakly. It means that the boundary conditions' choice does not imply the choice of the vectorial space as in continuous Galerkin. The discretized space is always the same, only some terms in the weak formulation have in case of need to be changed. For this reason, the match of Discontinuous Galerkin and Dubiner Basis results to be particularly successful. \\
    
    \noindent To conclude, we refer to \cite{sherwin} for the transformation and so the definition of the Dubiner Basis with tetrahedra, thus in dimension $n=3$.
    
    \subsection{Implementation}\label{subsection_implementation}
    Some of the codes that we used were written by professor Antonietti as eval\_jacobi\_polynomial.m, used to evaluate the Jacobi polynomial $J_n^{\alpha,\beta}$ in the vector z, basis\_legendre\_dubiner.m, that generates the basis functions of legendre Qp--Pp on the reference element [-1,1] x [-1,1], and evalshape\_tria\_dubiner.m, it creates the Dubiner basis (dphiq) and its gradient (Grad) on the volume terms and on the edges (B\_edge and G\_edge). So we used these ones as starting point for the functions that we needed.\newline
    Other simply changes in the code of (\ref{marta}), that we won't cover in this relation, were made in order to call the right function according to basis we have chosen (Dubiner or FEM).
    (METTERE CODICE DI QUESTE FUNZIONI?)
    \subsubsection{matrix2D\_dubiner.m}
    At first we implemented a function that assembles the matrices that we have defined in (\ref{matrices}):
    METTERE CODICE?
\subsubsection{dubiner\_to\_fem.m and fem\_to\_dubiner.m}
One of the many advantages of the FEM basis is that the evaluation of a basis function in a point of the mesh is equal to 1 only if that point is the one associated to the basis, 0 otherwise:
	\begin{equation} \label{ref1}
	\psi_i(x_j)=\delta_{ij}
	\end{equation}
	This property cannot be satisfied by Dubiner basis (although other good properties hold in this case, for instance regularity and especially \emph{orthogonality}). Indeed these basis have not localized support and they are neither normalized on the mesh edges. This means that the coefficients of the solution of the Dubiner system are \emph{not} the evaluation over the mesh points of the discretized function itself. They have a completely different meaning, they are now \emph{modal} values instead of being \emph{nodal}.
	For this reason we introduced two new functions that best transform the coefficients of the solution w.r.t. FEM basis to the coefficients w.r.t. Dubiner basis and viceversa.\vspace{5mm}
	
	\noindent Consider an element $\mathcal{K}\in \tau_h$ and $\{\psi_{i}\}_{i=1}^{p}$,$\{\phi_{j}\}_{j=1}^{q}$ as, respectively, the set of FEM functions and the set of Dubiner functions with support in $\mathcal{K}$. In addition, consider as $\{\hat{u}_i\}_{i=1}^p$,$\{\tilde{u}_j\}_{j=1}^q$ as, respectively, the FEM and Dubiner coefficients of the solution. \vspace{5mm}
	
	\noindent Let us start from the transformation to the FEM coefficients. We now exploit the property \ref{ref1}, i.e. the coefficient $\hat{u}_i$ is nothing else but the evaluation of $u_h$ on the i-th mesh point, then: 
	\begin{equation} \label{ref3}
	\hat{u}_i = \sum_{j=1}^q \tilde{u}_j\phi_j(x_i)
	\end{equation}
	where $x_i$ is the point associated to the $\psi_i$ basis function. \vspace{5mm}
	
	\noindent Instead, to compute the coefficients conversely, we need to exploit the fact that the Dubiner Basis are $L^2$-orthonormal (property obtained thanks to \ref{jac}). We then need to compute a $L^2$ scalar product between the FEM discretized function and each Dubiner basis function. That means:
	\begin{equation}\label{ref4}
	\tilde{u}_j = \int_\mathcal{K} u_h(x) \phi_j(x) \,dx = \int_{\mathcal{K}} \sum_{i=1}^p \hat{u}_i\psi_i(x) \phi_j(x) \,dx = \sum_{i=1}^p \Big(\int_{\mathcal{K}}\psi_i(x)\phi_j(x)\,dx \Big) \hat{u}_i
	\end{equation}
	
	\vspace{5mm}
	\noindent If the Dubiner functions are chosen as Galerkin basis, both the transformations are needed for the code implementation. Formula \ref{ref3} is needed to plot and compute errors after the resolution of the system (otherwise solely Dubiner coefficients are useless). Formula \ref{ref4} is instead needed to convert the FEM initial data $u_0$ into a vector of Dubiner coefficients before the resolution of the system.
	
	\vspace{5mm}
	\noindent For the sake of both simplicity and logic, we have decided to implement these transformations only from $P_n$ to $D_n$, $n=1,2,3$ and viceversa. Indeed, from one side, the degree of FEM is here less "important" since it contributes only to the number of points to which evaluate the computed solution. Then, increasing $n$ for P does not substantially improve the quality of the solution. On the other side, choosing the same degree for P and D means having the same number of local nodes ($nln$). For this reason, both $p$ and $q$ are replaced with $nln$ in the code.	
	Finally, we see the first transformation associated to \ref{ref3}:
	\begin{verbatim}
	function [u0] = dubiner_to_fem (uh, femregion, Data)         
deg=sscanf(Data.fem(2:end),'%f');
s=0;

% define the coordinates of the degree of freedom in the reference triangle
% adn square
if (deg==1)  %D1
    a   = [-1; 1; -1];
    b   = [-1; -1; 1];
    
elseif (deg ==2) %D2
    a   = [-1; 0; 1; 1; -1; -1];
    b   = [-1; -1; -1; 0; 1; 0];

elseif (deg==3) %D3
    a   = [-1; -0.5; 0.5; 1; 1; 1; -1; -1; -1; 0];
    b   = [-1; -1; -1; -1; -0.5; 0.5; 1; 0.5; -0.5; -1/3];   
end

csi=(1+a).*(1-b)/4;
eta=(1+b)/2;

% evaluate the Dubiner basis on these points
for j=0:(deg)
    for i=0:(deg)
        if (i+j) <= deg
           s=s+1;
           [pi] = eval_jacobi_polynomial(i,0,0,a);
           [pj] = eval_jacobi_polynomial(j,2.*i+1,0,b);
           cij=sqrt((2.*i +1).*2.*(i+j+1)./4.^i);
           phi(1,:,s)=cij.*(2.^i).*((1-eta).^i).*pi.*pj;
        end
    end
end

u0 = zeros(femregion.ndof,1);

for ie = 1:femregion.ne
    
   index = (ie-1)*femregion.nln*ones(femregion.nln,1) + [1:femregion.nln]';
   
   for i = 1 : femregion.nln
       for j = 1: femregion.nln
         u0(index(i)) = u0(index(i)) +  uh(index(j))*phi(1,i,j);
       end
   end
    
end
\end{verbatim}
And the second one associated to \ref{ref4}:
\begin{verbatim}
function [u0] = fem_to_dubiner (uh, femregion, Data)

% quadrature nodes and weights for integrals
[nodes_1D, ~, nodes_2D, w_2D] = quadrature(Data.nqn);

% evaluation of shape functions on quadrature point both on FEM basis and
% Dubiner basis
[shape_basis] = basis_legendre_dubiner(femregion.fem);
[phi_dub, ~, ~, ~] = evalshape_tria_dubiner(shape_basis,nodes_2D, nodes_1D,
Data.nqn,femregion.nln);
[shape_basis] = basis_lagrange(append("P", femregion.fem(2)));
[phi_fem, ~, ~, ~] = evalshape(shape_basis,nodes_2D,nodes_1D,femregion.nln);

u0 = zeros(femregion.ndof,1);

for ie = 1:femregion.ne
    index = (ie-1)*femregion.nln*ones(femregion.nln,1) + [1:femregion.nln]';
    for i = 1 : femregion.nln
        for k = 1:length(w_2D) 
            uh_eval_k = 0;
            for j = 1:femregion.nln
                uh_eval_k = uh_eval_k + uh(index(j))*phi_fem(1,k,j);
            end
            u0(index(i)) = u0(index(i)) + uh_eval_k*phi_dub(1,k,i).*w_2D(k);
        end
    end    
end
\end{verbatim}

\section{Totally discretized numerical methods}
The system is time dependent, so we divide the interval (0,T] into K subintervals $(t^k,t^k+1]$ of length $\Delta t$ such that $t^k=k \Delta t \qquad \forall k=0,\cdots,K-1$ and so we consider $V_m^k\approx V_m(t^k)$.
\subsection{Semi implicit method}
At first, we consider an implicit method for the time discretization, meanwhile for the non linear contribution of $I^{ion}$ a semi implicit method, indeed we think about it as a cubic function, so we treat explicitely the quadratic term and the rest of the terms implicitly, that is:
\begin{equation*}
I_{ion}=k(V_m^k-a)(V_m^k-1)V_m^{k+1}+w^{k+1}
\end{equation*}
Moreover in the ODE of the model \ref{algebraic}, we evaluate $w$ implicitely and we consider $V_m$ at the previous temporal step $t^k$:
\begin{equation*}
M \frac{w^{k+1}-w^k}{\Delta t}=\epsilon M (V_m^k-\gamma w^{k+1})
\end{equation*}
\vspace{5mm}
Therefore we obtain:
\begin{equation}
 \begin{gathered}
 \chi_mC_m \begin{bmatrix}M &-M \\ -M & M \end{bmatrix}
	\begin{bmatrix}\bm{\frac{\phi_i^{k+1}-\phi_i^{k}}{\Delta t}} \\ \bm{\frac{\phi_e^{k+1}-\phi_e^{k}}{\Delta t}}  \end{bmatrix}
	 + \begin{bmatrix}A_i & 0 \\ 0 & A_e \end{bmatrix}
	 \begin{bmatrix}\bm{\phi_i^{k+1}} \\ \bm{\phi_e^{k+1}} \end{bmatrix} +\\
	   \begin{bmatrix}C(V_m^h) & -C(V_m^h) \\ -C(V_m^h) & C(V_m^h) \end{bmatrix} 
	   \begin{bmatrix} \bm{\phi_i^{k+1}} \\ \bm{\phi_e^{k+1}}  \end{bmatrix} 
	   +\chi_m \begin{bmatrix}M & 0 \\ 0 & -M \end{bmatrix} 
	   	\begin{bmatrix}w^{k+1} \\ w^{k+1} \end{bmatrix} = 
	   	\begin{bmatrix} F_i^{k+1} \\ F_e^{k+1}\end{bmatrix}\\
	   M \frac{w^{k+1}-w^k}{\Delta t}=\epsilon M (V_m^k-\gamma w^{k+1})
\end{gathered}
\end{equation}
where $V_m^k=\phi_i^k-\phi_e^k$.\newline
\vspace{5mm}
We can rewrite it as:
\begin{equation}
\begin{gathered}
\left( \frac{\chi_m C_m}{\Delta t} \begin{bmatrix} M & -M \\ -M & M \end{bmatrix} + \begin{bmatrix} A_i & 0 \\ 0 & A_e \end{bmatrix} + 
\begin{bmatrix}
C(V_m^k) & -C(V_m^k) \\ -C(V_m^k) & C(V_m^k)
\end{bmatrix}\right)
\begin{bmatrix} \bm{\phi_i^{k+1}} \\ \bm{\phi_e^{k+1}} \end{bmatrix} = 
\\
\begin{bmatrix} F_i^{k+1} \\ F_e^{k+1} \end{bmatrix} 
- \chi_m \begin{bmatrix}M & 0 \\ 0 & -M \end{bmatrix}
\begin{bmatrix} w^{k+1} \\ w^{k+1} \end{bmatrix}
+ \frac{\chi_m C_m}{\Delta t} \begin{bmatrix}M & 0 \\ 0 & -M\end{bmatrix}
\begin{bmatrix} V_m^{k} \\ V_m^{k} \end{bmatrix}
\end{gathered}
\end{equation}
\begin{equation}
(\frac{1}{\Delta t}+\epsilon \gamma)M w^{k+1}=\epsilon M V_m^k+\frac{M}{\Delta t} w^k
\end{equation}
\vspace{5mm}
Defining:
\begin{itemize}
\item $B=\frac{\chi_m C_m}{\Delta t} \begin{bmatrix} M &-M \\-M & M \end{bmatrix}+\begin{bmatrix} A_i & 0 \\ 0 & A_e \end{bmatrix}$
\item $C_{nl}(V_m^k)=\begin{bmatrix}C(V_m^h) & -C(V_m^h) \\ -C(V_m^h) & C(V_m^h) \end{bmatrix}$
\item $r^{k+1}=\begin{bmatrix} F_i^{k+1} \\ F_e^{k+1}\end{bmatrix}-\chi_m \begin{bmatrix}M & 0 \\ 0 & -M \end{bmatrix} \begin{bmatrix}w^{k+1} \\ w^{k+1} \end{bmatrix}+\frac{\chi_m C_m}{\Delta t} \begin{bmatrix} M &-M \\-M & M \end{bmatrix} \begin{bmatrix} \phi_i^k \\ \phi_e^k \end{bmatrix}$
\end{itemize}
We get the final system:\newline
Find $\Phi^{k+1}=[\phi_i^{k+1} \phi_e^{k+1}]^T$ and $w^{k+1}$ $\forall k=0,\cdots,T-1$ such that:
\begin{equation}
\begin{cases}
(\frac{1}{\Delta t}+\epsilon \gamma)M w^{k+1}=\epsilon M V_m^k+\frac{M}{\Delta t} w^k \\
(B+C_{nl}(\Phi^k)) \Phi^{k+1}=r^{k+1}
\end{cases}
\end{equation} 

\subsubsection{Results}
Same method for the time discretization used in (\ref{marta})
PLOT

\subsection{Quasi implicit operator splitting}
The main characteristic of an operator splitting is to divide the problem into two different systems with two different operators, such that $L(u)=L_1(u)+L_2(u)$, and, starting from $u^n$, we find $\tilde(u^{n+1})$ through the first system, then the solution $u^{n+1}$ through the second one.
In the Quasi implicit operator splitting we treat in a explicit way any term, except the $V_m^n$ in the non linear term.\newline
Find $\tilde{V}_m^{k+1}$ and $w^{k+1}$ such that:
\begin{equation}
\begin{cases}
\chi_m C_m M \frac{\tilde{V}_m^{k+1}-V_m^k}{\Delta t} +  C(V_m^k) V_m^{k+1} + \chi_m M w^{k+1}= 0\\
\frac{w^{k+1} - w^k}{\Delta t} = \epsilon (V_m^{k+1}-\gamma w^{k+1})
\end{cases}
\end{equation}
Find $V_m^{k+1}$ such that:
\begin{equation}
\begin{cases}
\chi_m C_m M \frac{V_m^{k+1}-\tilde{V}_m^{k+1}}{\Delta t} + A_i \phi_i^{k+1}= F_i^{k+1}\\
- \chi_m C_m M \frac{V_m^{k+1}-\tilde{V}_m^{k+1}}{\Delta t} + A_e \phi_e^{k+1}= F_e^{k+1}
\end{cases}
\end{equation}
Put into a unique system:
\begin{equation}\label{Quasi}
\begin{cases}
\chi_m C_m M \frac{V_m^{k+1}-V_m^{k}}{\Delta t} + C(V_m^k) V_m^{k+1} + \chi_m M w^{k+1} + A_i \phi_i ^{k+1} = F_i^{k+1} \\
\chi_m C_m M \frac{V_m^{k+1}-V_m^{k}}{\Delta t} +  C(V_m^k) V_m^{k+1} + \chi_m M w^{k+1} - A_e \phi_e ^{k+1} =  -F_e^{k+1} \\
\frac{w^{k+1}-w^{k}}{\Delta t} = \epsilon(V_m^{k+1}-\gamma w^{k+1})
\end{cases}
\end{equation}
Defining:
\begin{itemize}
\item $Q_k = \frac{\chi_m C_m}{\Delta t}M + C(V_m^k) + \frac{\epsilon\chi_m \Delta t}{1 + \epsilon \gamma \Delta t} M$ \\
\item $R_k = \frac{\chi_mC_m}{\Delta t}MV_m^k - \frac{\chi_m}{1+\epsilon\gamma\Delta t}M w^k$
\end{itemize}
The equations in the system \ref{Quasi} can be written as:
\begin{enumerate}
\item
\begin{equation*}
\begin{gathered}
\chi_m C_m M \frac{	\phi_i^{k+1}-\phi_e^{k+1}-V_m^{k}}{\Delta t} +  C(V_m^k) (\phi_i^{k+1}-\phi_e^{k+1}) + \chi_m M \left(\frac{w^k + \epsilon \Delta t (\phi_i^{k+1}-\phi_e^{k+1})}{1+\epsilon \gamma \Delta t}   \right)+\\
+ A_i \phi_i ^{k+1} = F_i^{k+1} \\
\Rightarrow \quad (Q_k + A_i) \phi_i^{k+1} - Q_k \phi_e^{k+1} =R_k +  F_i^{k+1}
\end{gathered}
\end{equation*}
\item
\begin{equation*}
\begin{gathered}
\chi_m C_m M \frac{	\phi_i^{k+1}-\phi_e^{k+1}-V_m^{k}}{\Delta t} + \cdot C(V_m^k) (\phi_i^{k+1}-\phi_e^{k+1}) + \chi_m M \left(\frac{w^k + \epsilon \Delta t (\phi_i^{k+1}-\phi_e^{k+1})}{1+\epsilon \gamma \Delta t}   \right) +\\
- A_e \phi_e ^{k+1} = -F_e^{k+1} \\ \\
\Rightarrow \quad Q_k \phi_i^{k+1} - (Q_k+A_e) \phi_e^{k+1} =R_k - F_e^{k+1}
\end{gathered}
\end{equation*}
\item 
\begin{equation*}
w^{k+1} = \frac{w^k + \epsilon \Delta t (\phi_i^{k+1}-\phi_e^{k+1})}{1+\epsilon \gamma \Delta t}
\end{equation*}
\end{enumerate}
The final system becomes:\newline
Find $\Phi^{k+1}=[\phi_i^{k+1} \phi_e^{k+1}]^T$ and $w^{k+1} \qquad \forall k=0, \cdots, K-1$ such that:
\begin{equation}
\quad
\begin{cases}
\left(
\begin{bmatrix} Q_k & -Q_k \\ Q_k & -Q_k \end{bmatrix} + 
\begin{bmatrix} A_i & 0 \\ 0 & -A_e\end{bmatrix}
\right)
\begin{bmatrix}
\bm{\phi_i^{k+1}} \\ \bm{\phi_e^{k+1}}
\end{bmatrix}
= \begin{bmatrix} R_k \\ R_k \end{bmatrix} + \begin{bmatrix} F_i^{k+1} \\  -F_e^{k+1}\end{bmatrix} \\ \\
w^{k+1} = \frac{\displaystyle w^k + \epsilon \Delta t (\phi_i^{k+1}-\phi_e^{k+1})}{\displaystyle 1+\epsilon \gamma \Delta t}
\end{cases}
\end{equation}

\subsubsection{Implementation}
\begin{verbatim}
 ZERO = sparse(ll,ll);
        
    for t=dt:dt:T
        
        [C] = assemble_nonlinear(femregion,Data,Vm0);
         Q  = (ChiM*Cm/dt)*M + C - (epsilon*ChiM*dt)/(1+epsilon*gamma*dt)*M;
         R  = (ChiM*Cm/dt)*M*Vm0 - (ChiM)/(1+epsilon*gamma*dt)*M*w0;
    
        fi = assemble_rhs_i(femregion,neighbour,Data,t);
        fe = assemble_rhs_e(femregion,neighbour,Data,t);
        f1 = cat(1, fi, -fe);
    
        B = [Q, -Q; Q, -Q] + [Ai, ZERO; ZERO, -Ae];
        r = [R;R] + f1;
        
        u1 = B \ r; 
        
        Vm1 = u1(1:ll)-u1(ll+1:end);

        w1 = (w0 + epsilon*dt*Vm1)/(1+epsilon*gamma*dt);
    
        f0 = f1;
        Vm0 = u1(1:ll) - u1(ll+1:end);
        u0 = u1;
        w0 = w1;
    end
\end{verbatim}
\subsubsection{Results}
PLOT

\subsection{Godunov operator splitting}
Another type of operator splitting is the Gudonov operator splitting where in this case the first system is written in an explicit way, meanwhile the second implicitly. Indeed:\newline
Find $\tilde{V}_m^{k+1}$ and $w^{k+1}$ such that:
\begin{equation*}
\begin{cases}
\chi_m C_m M\frac{\hat{V}_m^{k+1} - V_m^k}{\Delta t} + C(V_m^k)V_m^k +\chi_mMw^k = 0\\
\frac{w^{k+1}-w^k}{\Delta t} = \epsilon (V_m^k - \gamma w^k)
\end{cases}
\end{equation*}
Find $V_m^{k+1}$ such that:
\begin{equation*}
\begin{cases}
\chi_m C_m M\frac{V_m^{k+1} - \hat{V}_m^{k+1}}{\Delta t} + A_i\phi_i^{k+1} = F_i^{k+1} \\
-\chi_m C_m M\frac{V_m^{k+1} - \hat{V}_m^{k+1}}{\Delta t} + A_e\phi_e^{k+1} = F_e^{k+1}
\end{cases}
\end{equation*}
Put in a unique system:
\begin{equation}\label{Godunov}
\begin{cases}
\chi_m C_m M\frac{V_m^{k+1} - V_m^k}{\Delta t} + C(V_m^k)V_m^k +\chi_mMw^k + A_i \phi_i^{k+1}= F_i^{k+1} \\
\chi_m C_m M\frac{V_m^{k+1} - V_m^k}{\Delta t} + C(V_m^k)V_m^k +\chi_mMw^k - A_e \Phi_e^{k+1}= -F_e^{k+1} \\
w^{k+1} = (1-\epsilon \gamma \Delta t) w^k + \epsilon \Delta t V_m^k
\end{cases}
\end{equation}
The equations in the system \ref{Godunov} can be written as:
\begin{equation*}
\begin{cases}
\left( \frac{\chi_m C_m}{\Delta t} M + A_i \right ) \phi_i^{k+1} - \frac{\chi_m C_m}{\Delta t} M \phi_e^{k+1} = F_i^{k+1} - \chi_m M w^k + \left( \frac{\chi_m C_m}{\Delta t} M- C(V_m^k)\right) V_m^k\\
\frac{\chi_m C_m}{\Delta t} M  \phi_i^{n+1} - \left(\frac{\chi_m C_m}{\Delta t} M + A_e \right) \phi_e^{k+1} =  -F_e^{k+1} - \chi_m M w^k + \left( \frac{\chi_m C_m}{\Delta t} M- C(V_m^k)\right) V_m^k \\
w^{k+1} = (1-\epsilon \gamma \Delta t) w^k + \epsilon \Delta tV_m^k
\end{cases}
\end{equation*}

In the end, find $\Phi^{k+1}=[\phi_i^{k+1} \phi_e^{k+1}]^T$ and $w^{k+1} \qquad \forall k=0, \cdots, K-1$ such that: 
\begin{equation}
\begin{cases}
\begin{gathered}
\left(
	\frac{\chi_m C_m}{\Delta t} \begin{bmatrix}M & -M \\ M & -M\end{bmatrix}
	+ \begin{bmatrix} A_i & 0 \\ 0 & -A_e \end{bmatrix}
	\right) \begin{bmatrix} \bm{\phi_i^{k+1}} \\ \bm{\phi_e^{k+1}}  \end{bmatrix} =
	\begin{bmatrix} F_i^{k+1} \\ -F_e^{k+1} \end{bmatrix} + \\ -
	\chi_m\begin{bmatrix} M & 0 \\ 0 & M \end{bmatrix} \begin{bmatrix} w^k \\ w^k \end{bmatrix} +
	\left(\frac{\chi_mC_m}{\Delta t}\begin{bmatrix} M & 0 \\ 0 & M \end{bmatrix}
	- \begin{bmatrix} C(V_m^k) & 0 \\ 0 & C(V_m^k)\end{bmatrix} 
	\right) \begin{bmatrix} V_m^k \\ V_m^k \end {bmatrix}
	 \end{gathered} \\ \\
	w^{k+1} = (1-\epsilon \gamma \Delta t) w^k + \epsilon \Delta tV_m^k
\end{cases}
\end{equation}
\subsubsection{Implementation}
\begin{verbatim}
ZERO = sparse(ll,ll);
    MASS = (ChiM*Cm/dt)*[M, -M; M -M];
    MASSW = ChiM*[M, ZERO; ZERO, M];
    
    for t=dt:dt:T
        
        fi = assemble_rhs_i(femregion,neighbour,Data,t);
        fe = assemble_rhs_e(femregion,neighbour,Data,t);
        f1 = cat(1, fi, -fe);
    
        [C] = assemble_nonlinear(femregion,Data,Vm0);
        
        w1 = (1 -epsilon*gamma*dt)*w0 + epsilon*dt*Vm0;
        B = MASS + [Ai, ZERO; ZERO, -Ae];
        r = -MASSW*[w0;w0] + ((Cm/dt)*MASSW - [C, ZERO; ZERO, C])
        *[Vm0;Vm0] + f1;
 
        Vm0 = u1(1:ll) - u1(ll+1:end);
        u0 = u1;
        w0 = w1;
    end
    \end{verbatim}
    \subsubsection{Results}
    PLOT
    \newpage
    \printbibliography

\end{document}

    
	
    
    
    
    
    
    
    
    
    
    
    
    \newpage
    \printbibliography

\end{document}
