% !TEX encoding = UTF-8
% !TEX program = pdflatex
\documentclass[a4paper]{article}
\usepackage[a4paper,top=3cm,bottom=4cm,left=3.8cm,right=3.8cm]{geometry}
\usepackage[T1]{fontenc}
\usepackage[utf8]{inputenc}
\usepackage[italian]{babel}
\usepackage{graphicx}
\usepackage{wrapfig}
\usepackage{comment}
\usepackage{amsmath}
\usepackage{amsfonts}
\usepackage{amsthm}
\usepackage{enumerate}
\usepackage{enumitem}
\usepackage{physics}
\usepackage{bm}
  
\theoremstyle{definition}  
\newtheorem*{defn}{Definizione}  
\newtheorem*{prop}{Proposizione} 
\newtheorem*{oss}{Osservazione}
\newtheorem {prob}{Problema}


\SetLabelAlign{Center}{\hfil#1\hfil}

\newenvironment{rcases}
{\left.\begin{aligned}}
	{\end{aligned}\right\rbrace}

\begin{document}
\title{Formulario per il progetto NAPDE}
\author{Matteo Calafà e Federica Botta}
\maketitle
\vspace{5mm}
\section{Monodominio}
\vspace{5mm}
\subsection{Modelli analitici}
\vspace{5mm}
\paragraph{Modello del monodominio}
	\begin{equation}
	\begin{cases}
	\chi_m C_m\pdv{V_m}{t} - \nabla \cdot (\Sigma \nabla V_m) + \chi_m I_{ion}(V_m,w) = I^{ext}    & \text{in } \Omega_{mus} \cross (0,T]
	\\
	\pdv{w}{t} = g(V\ped{m}, w)  & \text{in } \Omega_{mus} \cross (0,T]
	\\
	\Sigma\nabla V\ped{m} \cdot n = b   & \text{on } \partial \Omega_{mus} \cross (0,T]
	\end{cases}
	\end{equation}
	\vspace{4mm}\newline
	dove le incognite sono:
	\begin{itemize}
		\item $V_m$ = $\Phi_i - \Phi_e$ (differenza tra potenziale interno e esterno)
		\item $w$  ("\emph{gating variable}")
	\end{itemize}
    e sono date le costanti : $\chi_m, C_m, \Sigma$



\paragraph{Modello di FitzHugh-Nagumo}
\begin{equation}
\begin{gathered}
I_{ion}(V_m, w) = -kV_m(V_m-a)(V_m-1) -w
\\
g(V_m,w) = \epsilon(V_m -\gamma w)
\end{gathered}
\end{equation}

\vspace{5mm}

\subsection{Modello numerico semi-discretizzato}

	\vspace{5mm}
	 \begin{equation}
	 \begin{rcases}
	 V_{ij} &= \int_{\Omega}\nabla\varphi_j \cdot \nabla \varphi_i 
	 \\ I_{i,j}^T &= \sum_{F \in F_h^I} \int_{F} \{\{\nabla\varphi_j\}\} \cdot [[\varphi_i]] 
	\\ I_{i,j} &= \sum_{F \in F_h^I} \int_{F} [[\varphi_j]] \cdot \{\{\nabla \varphi_i\}\}
	 \\S_{i,j} &= \sum_{F \in F_h^I} \int_{F} \gamma [[\varphi_j]] \cdot [[\varphi_i]]]
	 \end{rcases}
	 \quad A = \Sigma(V -I^T - \delta I + S)
	\end{equation}
	\begin{equation}
	M_{ij} = \sum_{K \in \tau_h}\int_K
	\varphi_j\varphi_i
	\end{equation}
	\begin{equation}
	C(u_h)_{ij} = - \sum_{K \in \tau_h} \int_K \chi_m k(u_h-1)(u_h-a)\varphi_j\varphi_i
	\end{equation}
	\begin{equation}
	F_i = \int_{\Omega} f\varphi_i - \sum_{F \in F_h^B} \int_F b\varphi_i
	\end{equation}
	
	\vspace{3mm}
\paragraph{Problema semi-discretizzato} 

\begin{equation*}
\begin{gathered}
\{\varphi_j\}_{j=1}^{N_h} \text{ base di } V_h^p = \{v_h \in L^2 : v_h|_K \in \mathbb{P}^{p_k}(K) \quad p_k \leq p \quad \forall K \in \tau_h \}
\\
u_h(t) = \sum_{j=1}^{N_h} u_j(t)\varphi_j, \quad w_h(t) = \sum_{j=1}^{N_h}w_j(t)\varphi_j
\end{gathered}
\end{equation*}
\vspace{3mm}
\begin{equation}
\Rightarrow \quad  \begin{gathered}\boxed{\chi_mC_m M \dot{u} +  A u + C(u_h) u -\chi_mMw=F}
	\\
	\boxed{\dot{w} = \epsilon(u-\gamma w)}
	\end{gathered}
\end{equation}
\vspace{5mm}
\subsection{Modello numerico totalmente discretizzato ("$\bm{\theta-method}$")}
\vspace{5mm}
\paragraph{Forma implicita $(\theta \in [0,1])$}
\begin{enumerate}
	\item
	\begin{equation}
	\begin{gathered}
	\chi_mC_m M \frac{u^{k+1}-u^k}{\Delta t} +  A \left(\theta u^{k+1} + (1-\theta)u^k \right) + C(u^k)\left(\theta u^{k+1} + (1-\theta)u^k \right) +
	\\ -\chi_mMw^{k+1}=\theta F^{k+1} + (1-\theta)F^k
	\end{gathered}
	\end{equation}
	\item
	\begin{equation}
	\frac{w^{k+1}- w^{k}}{\Delta t} = \epsilon(u^k-\gamma w^{k+1})
	\end{equation}
\end{enumerate}
\vspace{3mm}
\paragraph{Forma esplicita $(\theta \in [0,1])$}
\begin{enumerate}
	\item
	\begin{equation}
	\begin{gathered}
	\left[\chi_mC_mM+\theta \Delta t A + \theta \Delta t C(u^k) \right] \bm{u^{k+1}} = \theta\Delta t F^{k+1} + (1-\theta)\Delta tF^k +
	\\
	\left[\chi_mC_mM- (1-\theta)\Delta t A - (1-\theta)\Delta t C(u^k)\right] u^k +\chi_m\Delta t Mw^{k+1} 
	\end{gathered}
	\end{equation}
	\item
	\begin{equation}
	\left[1+ \epsilon \gamma \Delta t\right] \bm{w^{k+1}} = w^k + (\epsilon \Delta t) u^k
	\end{equation}
\end{enumerate}







\newpage

\section{Bidominio}
\vspace{5mm}
\subsection{Modelli analitici}
\vspace{5mm}
\paragraph{Modello del bidominio}
\begin{equation}
\begin{cases}
\chi_m C_m\pdv{V_m}{t} - \nabla \cdot (\Sigma_i \nabla \phi_i) + \chi_m I_{ion}(V_m,w) = I_i^{ext}    & \text{in } \Omega_{mus} \cross (0,T]
\\
-\chi_m C_m\pdv{V_m}{t} - \nabla \cdot (\Sigma_e \nabla \phi_e) - \chi_m I_{ion}(V_m,w) = -I_e^{ext}    & \text{in } \Omega_{mus} \cross (0,T]
\\
\pdv{w}{t} = g(V\ped{m}, w)  & \text{in } \Omega_{mus} \cross (0,T]
\\
\Sigma_i\nabla \phi_i \cdot n = b_i   & \text{on } \partial \Omega_{mus} \cross (0,T]
\\
\Sigma_e\nabla \phi_e \cdot n = b_e   & \text{on } \partial \Omega_{mus} \cross (0,T]
\end{cases}
\end{equation}
\vspace{4mm}\newline
dove le incognite sono:
\begin{itemize}
	\item $\phi_i, \phi_e$  $\quad (V_m = \phi_i-\phi_e$)
	\item $w$  ("\emph{gating variable}")
\end{itemize}
e sono date le costanti : $\chi_m, C_m, \Sigma_i, \Sigma_e$

\paragraph{Modello di FitzHugh-Nagumo}
\begin{equation}
\begin{gathered}
I_{ion}(V_m, w) = -kV_m(V_m-a)(V_m-1) -w
\\
g(V_m,w) = \epsilon(V_m -\gamma w)
\end{gathered}
\end{equation}

\vspace{5mm}

\subsection{Modello numerico semi-discretizzato}

\vspace{5mm}
\begin{equation}
\begin{rcases}
V_{ij} &= \int_{\Omega}\nabla\varphi_j \cdot \nabla \varphi_i 
\\ I_{i,j}^T &= \sum_{F \in F_h^I} \int_{F} \{\{\nabla\varphi_j\}\} \cdot [[\varphi_i]] 
\\ I_{i,j} &= \sum_{F \in F_h^I} \int_{F} [[\varphi_j]] \cdot \{\{\nabla \varphi_i\}\}
\\S_{i,j} &= \sum_{F \in F_h^I} \int_{F} \gamma [[\varphi_j]] \cdot [[\varphi_i]]]
\end{rcases}
\begin{gathered}
\quad A = (V -I^T - \theta I + S)\\
\quad A_i = \Sigma_iA \\
\quad A_e = \Sigma_eA
\end{gathered}
\end{equation}
\begin{equation}
M_{ij} = \sum_{K \in \tau_h}\int_K
\varphi_j\varphi_i
\end{equation}
\begin{equation}
C(u_h)_{ij} = - \sum_{K \in \tau_h} \int_K \chi_m k(u_h-1)(u_h-a)\varphi_j\varphi_i
\end{equation}
\begin{equation}
\begin{gathered}
F_{i,k} = \int_{\Omega} I_i^{ext}\varphi_k - \sum_{F \in F_h^B} \int_F b_i\varphi_k
\\
F_{e,k} = - \int_{\Omega} I_e^{ext}\varphi_k - \sum_{F \in F_h^B} \int_F b_e\varphi_k
\end{gathered}
\end{equation}


	\vspace{3mm}
\paragraph{Problema semi-discretizzato} 

\begin{equation*}
\begin{gathered}
\{\varphi_j\}_{j=1}^{N_h} \text{ base di } V_h^k = \{v_h \in L^2 : v_h|_\mathcal{K} \in \mathbb{P}^{k}(\mathcal{K})  \quad \forall \mathcal{K} \in \tau_h \}
\\
\Phi_h(t) = \begin{bmatrix} \Phi_i^h(t) \\ \Phi_e^h(t) \end{bmatrix} = \begin {bmatrix}\sum_{j=1}^{N_h} \Phi_{i,j}(t)\varphi_j \\ \sum_{j=1}^{N_h} \Phi_{e,j}(t)\varphi_j \end{bmatrix},
 \quad w_h(t) = \sum_{j=1}^{N_h}w_j(t)\varphi_j
\end{gathered}
\end{equation*}
\vspace{3mm}
\begin{equation}
\Rightarrow \quad
\boxed{
\begin{gathered}
 \chi_mC_m \begin{bmatrix}M &-M \\ -M & M \end{bmatrix}
	\begin{bmatrix}\bm{\dot{\Phi}_i^h(t)} \\ \bm{\dot{\Phi}_e^h(t)} \end{bmatrix}
	 + \begin{bmatrix}A_i & 0 \\ 0 & A_e \end{bmatrix}
	 \begin{bmatrix}\bm{\Phi_i^h(t)} \\ \bm{\Phi_e^h(t)} \end{bmatrix} +\\
	   \begin{bmatrix}C(V_m^h) & -C(V_m^h) \\ -C(V_m^h) & C(V_m^h) \end{bmatrix} 
	   \begin{bmatrix} \bm{\Phi_i^h(t)} \\ \bm{\Phi_e^h(t)}  \end{bmatrix} 
	   -\chi_m \begin{bmatrix}M & 0 \\ 0 & -M \end{bmatrix} 
	   	\begin{bmatrix}w_h(t) \\ w_h(t) \end{bmatrix} = 
	   	\begin{bmatrix} F_i^h \\ F_e^h\end{bmatrix}
\end{gathered}
}
\end{equation}
\begin{equation}
\boxed{\dot{w}_h(t) = \epsilon(V_m^h(t) - \gamma w_h(t))}
\end{equation}
\vspace{5mm}

\subsection{Modelli numerici totalmente discretizzati}
\vspace{4mm}
\subsubsection{Metodo semi-implicito}
\paragraph{Forma implicita}
\begin{equation}
\begin{gathered}
\chi_mC_m \begin{bmatrix}M &-M \\ -M & M \end{bmatrix}
\begin{bmatrix}\frac{\Phi_i^{k+1}-\Phi_i^k}{\Delta t} \\ \frac{\Phi_e^{k+1}-\Phi_e^k}{\Delta t} \end{bmatrix}
+ \begin{bmatrix}A_i & 0 \\ 0 & A_e \end{bmatrix}
\begin{bmatrix}\Phi_i^{k+1} \\ \Phi_e^{k+1} \end{bmatrix} + \\
 \begin{bmatrix}C(V_m^k) & -C(V_m^k) \\ -C(V_m^k) & C(V_m^k) \end{bmatrix} 
\begin{bmatrix} \Phi_i^{k+1} \\ \Phi_e^{k+1}  \end{bmatrix} 
-\chi_m \begin{bmatrix}M & 0 \\ 0 & -M \end{bmatrix} 
\begin{bmatrix}w^{k+1} \\ w^{k+1} \end{bmatrix} = 
\begin{bmatrix} F_i^{k+1} \\ F_e^{k+1}\end{bmatrix}
\end{gathered}
\end{equation}
\vspace{5mm}
\begin{equation}
\frac{w^{k+1}-w^k}{\Delta t} = \epsilon (V_m^k - \gamma w^{k+1})
\end{equation}

\paragraph{Forma esplicita}
\begin{equation}
\begin{gathered}
\left( \frac{\chi_m C_m}{\Delta t} \begin{bmatrix} M & -M \\ -M & M \end{bmatrix} + \begin{bmatrix} A_i & 0 \\ 0 & A_e \end{bmatrix} + 
\begin{bmatrix}
C(V_m^k) & -C(V_m^k) \\ -C(V_m^k) & C(V_m^k)
\end{bmatrix}\right)
\begin{bmatrix} \bm{\Phi_i^{k+1}} \\ \bm{\Phi_e^{k+1}} \end{bmatrix} = 
\\
\begin{bmatrix} F_i^{k+1} \\ F_e^{k+1} \end{bmatrix} 
+ \chi_m \begin{bmatrix}M & 0 \\ 0 & -M \end{bmatrix}
\begin{bmatrix} w^{k+1} \\ w^{k+1} \end{bmatrix}
+ \frac{\chi_m C_m}{\Delta t} \begin{bmatrix}M & -M \\ -M & M\end{bmatrix}
\begin{bmatrix} \Phi_i^{k} \\ \Phi_e^{k} \end{bmatrix}
\end{gathered}
\end{equation}

\vspace{5mm}
\begin{equation}
(1+\epsilon \gamma \Delta t)w^{k+1} = w^k + \epsilon \Delta t V_m^k
\end{equation}

\newpage
\subsubsection{Operator Splitting quasi-implicito}
\paragraph{Forma implicita}
\begin{enumerate}[label = \Roman*, align = Center]
	\item 
	\begin{equation}
	\begin{gathered}
	\chi_m C_m M \frac{\tilde{V}_m^{n+1}-V_m^n}{\Delta t} +  C(V_m^n) V_m^{n+1} - \chi_m M w^{n+1}= 0
	\\
	 \frac{w^{n+1} - w^n}{\Delta t} = \epsilon (V_m^{n+1}-\gamma w^{n+1})
	\end{gathered}
	\end{equation}
	\item 
	\begin{equation}
	\begin{gathered}
	\chi_m C_m M \frac{V_m^{n+1}-\tilde{V}_m^{n+1}}{\Delta t} + A_i \Phi_i^{n+1}= F_i^{n+1}
	\\
	- \chi_m C_m M \frac{V_m^{n+1}-\tilde{V}_m^{n+1}}{\Delta t} + A_e \Phi_e^{n+1}= F_e^{n+1}
	\end{gathered}
	\end{equation}
	
\end{enumerate}
\vspace{5mm}
\paragraph{Forma esplicita}
\begin{equation}
\vspace{5mm}
\begin{cases}
\chi_m C_m M \frac{V_m^{n+1}-V_m^{n}}{\Delta t} + C(V_m^n) V_m^{n+1} - \chi_m M w^{n+1} + A_i \Phi_i ^{n+1} = F_i^{n+1} \\
\chi_m C_m M \frac{V_m^{n+1}-V_m^{n}}{\Delta t} +  C(V_m^n) V_m^{n+1} - \chi_m M w^{n+1} - A_e \Phi_e ^{n+1} =  -F_e^{n+1} \\
\frac{w^{n+1}-w^{n}}{\Delta t} = \epsilon(V_m^{n+1}-\gamma w^{n+1})
\end{cases}
\end{equation}

 \begin{equation}
 \begin{gathered}
 \begin{aligned}
 & \bullet \quad Q_n := \frac{\chi_m C_m}{\Delta t}M + C(V_m^n) - \frac{\epsilon\chi_m \Delta t}{1 + \epsilon \gamma \Delta t} M \\
 & \bullet \quad R_n := \frac{\chi_mC_m}{\Delta t}MV_m^n + \frac{\chi_m}{1+\epsilon\gamma\Delta t}M w^n
 \end{aligned}
 \end{gathered}
\end{equation}

\vspace{5mm}
\begin{enumerate}
	\item 
	\begin{equation}
	\begin{gathered}
	\chi_m C_m M \frac{	\Phi_i^{n+1}-\Phi_e^{n+1}-V_m^{n}}{\Delta t} +  C(V_m^n) (\Phi_i^{n+1}-\Phi_e^{n+1}) + \\ - \chi_m M \left(\frac{w^n + \epsilon \Delta t (\Phi_i^{n+1}-\Phi_e^{n+1})}{1+\epsilon \gamma \Delta t}   \right) + A_i \Phi_i ^{n+1} = F_i^{n+1} \\ \\
     \Rightarrow \quad (Q_n + A_i) \Phi_i^{n+1} - Q_n \Phi_e^{n+1} =R_n +  F_i^{n+1}
	\end{gathered}
	\end{equation}
	
    \item 
	\begin{equation}
	\begin{gathered}
	\chi_m C_m M \frac{	\Phi_i^{n+1}-\Phi_e^{n+1}-V_m^{n}}{\Delta t} + \cdot C(V_m^n) (\Phi_i^{n+1}-\Phi_e^{n+1}) + \\ -  \chi_m M \left(\frac{w^n + \epsilon \Delta t (\Phi_i^{n+1}-\Phi_e^{n+1})}{1+\epsilon \gamma \Delta t}   \right) - A_e \Phi_e ^{n+1} = -F_e^{n+1} \\ \\
	\Rightarrow \quad Q_n \Phi_i^{n+1} - (Q_n+A_e) \Phi_e^{n+1} =R_n - F_e^{n+1}
	\end{gathered}
	\end{equation}
	\vspace{3mm}
	\item 
	\begin{equation}
	w^{n+1} = \frac{w^n + \epsilon \Delta t (\Phi_i^{n+1}-\Phi_e^{n+1})}{1+\epsilon \gamma \Delta t}
	\end{equation}
\end{enumerate}


\begin{equation}
\Rightarrow \quad
\begin{cases}
\left(
\begin{bmatrix} Q_n & -Q_n \\ Q_n & -Q_n \end{bmatrix} + 
\begin{bmatrix} A_i & 0 \\ 0 & -A_e\end{bmatrix}
\right)
\begin{bmatrix}
\bm{\Phi_i^{n+1}} \\ \bm{\Phi_e^{n+1}}
\end{bmatrix}
= \begin{bmatrix} R_n \\ R_n \end{bmatrix} + \begin{bmatrix} F_i^{n+1} \\  -F_e^{n+1}\end{bmatrix} \\ \\
Q_n = \frac{\displaystyle \chi_m C_m}{\displaystyle \Delta t}M +  \cdot C(V_m^n) - \frac{\displaystyle \epsilon\chi_m \Delta t}{\displaystyle 1 + \epsilon \gamma \Delta t}M \\ \\
R_n := \frac{\displaystyle \chi_mC_m}{\displaystyle \Delta t}MV_m^n + \frac{\displaystyle \chi_m}{\displaystyle 1+\epsilon\gamma\Delta t}M w^n \\ \\
w^{n+1} = \frac{\displaystyle w^n + \epsilon \Delta t (\Phi_i^{n+1}-\Phi_e^{n+1})}{\displaystyle 1+\epsilon \gamma \Delta t}
\end{cases}
\end{equation}


\vspace{5mm}
\subsubsection{Operator Splitting di Godunov}
\paragraph{Forma implicita}
\begin{enumerate}[label = \Roman*, align = Center]
\item 
   \begin{equation}
   \begin{gathered}
   \chi_m C_m M\frac{\hat{V}_m^{n+1} - V_m^n}{\Delta t} + C(V_m^n)V_m^n -\chi_mMw^n = 0
   \\
   \frac{w^{n+1}-w^n}{\Delta t} = \epsilon (V_m^n - \gamma w^n)
   \end{gathered}
   \end{equation}
\item 
   \begin{equation}
   \begin{gathered}
   \chi_m C_m M\frac{V_m^{n+1} - \hat{V}_m^{n+1}}{\Delta t} + A_i\Phi_i^{n+1} = F_i^{n+1} \\
   -\chi_m C_m M\frac{V_m^{n+1} - \hat{V}_m^{n+1}}{\Delta t} + A_e\Phi_e^{n+1} = F_e^{n+1}
   \end{gathered}
   \end{equation}
\end{enumerate}
\vspace{3mm}
\paragraph{Forma esplicita}

\begin{equation}
\begin{gathered}
\begin{cases}
\chi_m C_m M\frac{V_m^{n+1} - V_m^n}{\Delta t} + C(V_m^n)V_m^n -\chi_mMw^n + A_i \Phi_i^{n+1}= F_i^{n+1} \\
\chi_m C_m M\frac{V_m^{n+1} - V_m^n}{\Delta t} + C(V_m^n)V_m^n -\chi_mMw^n - A_e \Phi_e^{n+1}= -F_e^{n+1} \\
w^{n+1} = (1-\epsilon \gamma \Delta t) w^n + \epsilon \Delta t V_m^n
\end{cases}
\\ \\  
\Rightarrow
\begin{cases}
\left( \frac{\chi_m C_m}{\Delta t} M + A_i \right ) \Phi_i^{n+1} - \frac{\chi_m C_m}{\Delta t} M \Phi_e^{n+1} = F_i^{n+1} + \chi_m M w^n + \left( \frac{\chi_m C_m}{\Delta t} M- C(V_m^n)\right) V_m^n \\ 
\frac{\chi_m C_m}{\Delta t} M  \Phi_i^{n+1} - \left(\frac{\chi_m C_m}{\Delta t} M + A_e \right) \Phi_e^{n+1} =  -F_e^{n+1} + \chi_m M w^n + \left( \frac{\chi_m C_m}{\Delta t} M- C(V_m^n)\right) V_m^n \\
w^{n+1} = (1-\epsilon \gamma \Delta t) w^n + \epsilon \Delta tV_m^n
\end{cases}
\end{gathered} 
\end{equation}
\vspace{3mm}
\begin{equation}
\Rightarrow \quad 
\begin{cases}
\begin{gathered}
\left(
	\frac{\chi_m C_m}{\Delta t} \begin{bmatrix}M & -M \\ M & -M\end{bmatrix}
	+ \begin{bmatrix} A_i & 0 \\ 0 & -A_e \end{bmatrix}
	\right) \begin{bmatrix} \bm{\Phi_i^{n+1}} \\ \bm{\Phi_e^{n+1}}  \end{bmatrix} =
	\begin{bmatrix} F_i^{n+1} \\ -F_e^{n+1} \end{bmatrix} + \\
	\chi_m\begin{bmatrix} M & 0 \\ 0 & M \end{bmatrix} \begin{bmatrix} w^n \\ w^n \end{bmatrix} +
	\left(\frac{\chi_mC_m}{\Delta t}\begin{bmatrix} M & 0 \\ 0 & M \end{bmatrix}
	- \begin{bmatrix} C(V_m^n) & 0 \\ 0 & C(V_m^n)\end{bmatrix} 
	\right) \begin{bmatrix} V_m^n \\ V_m^n \end {bmatrix}
	 \end{gathered} \\ \\
	w^{n+1} = (1-\epsilon \gamma \Delta t) w^n + \epsilon \Delta tV_m^n
\end{cases}
\end{equation}
\newpage
\section{Basis}
\paragraph{Transformation}
	On the reference triangle
	\begin{equation}
	\hat{K}=\{ (\xi, \eta) : \xi, \eta \ge 0,	\xi+\eta \le 1 \}
	\end{equation}
	we consider the transformation between the reference square and the reference triangle given by
	\begin{equation}
	\xi:=\frac{(1+a)(1-b)}{4},  \eta:=\frac{(1+b)}{2}
	\end{equation}
	and the inverse transformation is
	\begin{equation}
	a:=\frac{2\xi-1+\eta}{1-\eta}=\frac{2\xi}{1-\eta}-1,	b:=2\eta-1
	\end{equation}

\paragraph{Dubiner Basis}
	\begin{equation}
	\begin{split}
	\phi_{ij}(\xi,\eta) :&= c_{ij}(1-b)^j J_i^{0,0}(a) J_j^{2i+1,0}(b)=
	\\&=c_{ij} 2^j (1-\eta)^j J_i^{0,0}(\frac{2\xi}{1-\eta}-1) J_j^{2i+1,0} (2\eta-1)
	\end{split}
	\end{equation}
	for i,j=1,....,p and i+j $\le$ p, where
	\begin{equation}
	c_{ij} := \sqrt{\frac{2(2i+1)(i+j+1)}{4^i}}
	\end{equation}
	and J$_i$$^{\alpha,\beta}$(.) is the i-th Jacobi polynomial
\paragraph{Jacobian Basis}
	J$_i$$^{\alpha,\beta}$(.) is orthogonal under the Jacobi weight w(x)=(1-x)$^\alpha$(1+x)$^\beta$ i.e. 
	\begin{equation} 
	\int_{-1}^{1}{(1-x)^\alpha(1+x)^\beta J_m^{\alpha,\beta} J_q^{\alpha,\beta}(x)dx}=\frac{2}{2m+1} \delta_{mq} 
	\end{equation}
	Evaluate the basis for a vector z of dimension n:
	\begin{equation}
	J_0^{\alpha,\beta}(z)=ones(1,n)
	\end{equation}
	\begin{equation}
	J_1^{\alpha,\beta}(z)=\frac{1}{2}(\alpha-\beta+(\alpha+\beta+2)*z);
	\end{equation}
	for n $\ge$ 2
	\newline
	\begin{equation}
	\begin{split}
	J_n^{\alpha,\beta}(z)=&\sum_{k=2}^{n} (\frac{(2k+\alpha+\beta-1)(\alpha^{2}-\beta^{2})}{2k(k+\alpha+\beta)(2k+\alpha+\beta-2)}+\frac{(2k+\alpha+\beta-2)(2k+\alpha+\beta-1)(2k+\alpha+\beta)}{2k(k+\alpha+\beta)(2k+\alpha+\beta-2)}) J_{k-1}^{\alpha,\beta}(z)
	\\&-\frac{2(k+\alpha-1)(k+\beta-1)(2k+\alpha+\beta)}{2k(k+\alpha+\beta)(2k+\alpha+\beta-2)} J_{k-2}^{\alpha,\beta}(z)
	\end{split}
	\end{equation}

\paragraph{Gradient of Dubiner Basis}
	for i=0 and j=0
	\begin{equation}
	\begin{split}
	&\phi_{00}^{\xi}(\xi,\eta)=0
	\\&\phi_{00}^{\eta}(\xi,\eta)=0
	\end{split}
	\end{equation}
	for i=0 and j$\neq$0
	\begin{equation}
	\begin{split}
	&\phi_{0j}^{\xi}=0
	\\&\phi_{0j}^{\eta}=c_{0j} (j+2) J_{j-1}^{2,1}(b)
	\end{split}
	\end{equation}
	for i$\neq$0 and j=0
	\begin{equation}
	\begin{split}
	&\phi_{i0}^{\xi}(\xi,\eta)=c_{i0} 2^{i} (1-\eta)^{i-1} (i+1) J_{i-1}^{1,1}(a)
	\\&\phi_{i0}^{\eta}(\xi,\eta)=c_{i0} 2^{i} (-i(1-\eta)^{i-1} J_{i}^{0,0}(a)+\xi(1-\eta)^{i-2}(i+1) J_{i-1}^{1,1}(a))
	\end{split}
	\end{equation}
	for i$\neq$0 and j$\neq$0
	\begin{equation}
	\begin{split}
	&\phi_{ij}^{\xi}(\xi,\eta)=c_{ij} 2^{i} (1-\eta)^{i-1} (i+1) J_{i-1}^{1,1}(a) J_{j}^{2i+1,0}(b)
	\\&\phi_{ij}^{\eta}(\xi,\eta)=c_{ij} 2^{i} (-i(1-\eta)^{i-1} J_{i}^{0,0}(a) J_{j}^{2i+1,0}(b)+\xi(1-\eta)^{i-2}(i+1) J_{i-1}^{1,1}(a) J_{j}^{2i+1,0}(b)
	\\&+(1-\eta)^{i}(2i+j+2) J_{i}^{0,0}(a) J_{j-1}^{2i+2,1}(b))
	\end{split}
	\end{equation}
\paragraph{Problems}
	One of the many advantages of the FEM basis is that the evaluation in a point of the mesh is a delta of Kronecker, so it's equal to 1 if that point is associated to the basis, 0 if not:
	\begin{equation}
	\hat{\psi}_i(x_j)=\delta_{ij}
	\end{equation}
	This property doesn't hold for the Dubiner basis, because its coefficients have another meaning and so they need a different evaluation.
	For this reason we introduce two new functions that transform the coefficients between FEM basis and Dubiner basis, such that:

	${\psi_{i}}$ is the FEM basis with coefficients $\hat{u_{i}}$ and ${\phi_{ij}}$ is the Dubiner basis with coefficients u$_i$
	
	To find the coefficients with respect to Dubiner basisi, first we evaluate both basis on the quadrature nodes, then
	\begin{equation}
	u_i=\hat{u_i} * (\sum_{j=1}^{nln} u_j \psi_j)=\int_{\hat{\Omega}} \hat{u_i} (\sum_{j=1}^{nln} u_j \psi_j) dx=\sum_{q=1}^{Nq} \hat{u_i} (\sum_{j=1}^{nln} u_j \psi_j(x_q)) w_q
	\end{equation}
	Instead, to compute the coefficients with respect FEM basis, we define the local degree of freedom of the reference square:
	
	-In 1D: a = [ -1; 1; -1] and b = [ -1; -1; 1]
	
	-In 2D: a = [-1; 0; 1; 1; -1; -1] and b = [-1; -1; -1; 0; 1; 0]
	
	-In 3D: a = [-1; -0.5; 0.5; 1; 1; 1; -1; -1; -1; 0] and b = [-1; -1; -1; -1; -0.5; 0.5; 1; 0.5; -0.5; -1/3]
	
	And the corresponding coordinates in the reference triangle using (37), using these we can calculate the Dubiner basis and so:
	\begin{equation}
	\hat{u_i}=\sum_{j=1}^{nln} u_i \phi_{ij}
	\end{equation}
	In conclusion, we use the first function to transform the initial condition with respect to Dubiner basis and the second one to convert the final solution to FEM to plot the solution and to compute the errors. 
\end{document}
