% !TEX encoding = UTF-8
% !TEX program = pdflatex
\documentclass[a4paper]{article}
\usepackage[a4paper,top=3cm,bottom=4cm,left=3.8cm,right=3.8cm]{geometry}
\usepackage[T1]{fontenc}
\usepackage[utf8]{inputenc}
\usepackage[italian]{babel}
\usepackage{graphicx}
\usepackage{wrapfig}
\usepackage{comment}
\usepackage{amsmath}
\usepackage{amsfonts}
\usepackage{amsthm}
\usepackage{enumerate}
\usepackage{physics}
\usepackage{bm}
  
\theoremstyle{definition}  
\newtheorem*{defn}{Definizione}  
\newtheorem*{prop}{Proposizione} 
\newtheorem*{oss}{Osservazione}
\newtheorem {prob}{Problema}


\newenvironment{rcases}
{\left.\begin{aligned}}
	{\end{aligned}\right\rbrace}

\begin{document}
\title{Formulario per il progetto NAPDE}
\author{Matteo Calafà}
\maketitle
\vspace{5mm}
\section{Monodominio}
\vspace{5mm}
\subsection{Modelli analitici}
\vspace{5mm}
\paragraph{Modello del monodominio}
	\begin{equation}
	\begin{cases}
	\chi_m C_m\pdv{V_m}{t} - \nabla \cdot (\Sigma \nabla V_m) + \chi_m I_{ion}(V_m,w) = I^{ext}    & \text{in } \Omega_{mus} \cross (0,T]
	\\
	\pdv{w}{t} = g(V\ped{m}, w)  & \text{in } \Omega_{mus} \cross (0,T]
	\\
	\Sigma\nabla V\ped{m} \cdot n = b   & \text{on } \partial \Omega_{mus} \cross (0,T]
	\end{cases}
	\end{equation}
	\vspace{4mm}\newline
	dove le incognite sono:
	\begin{itemize}
		\item $V_m$ = $\Phi_i - \Phi_e$ (differenza tra potenziale interno e esterno)
		\item $w$  ("\emph{gating variable}")
	\end{itemize}
    e sono date le costanti : $\chi_m, C_m, \Sigma$



\paragraph{Modello di FitzHugh-Nagumo}
\begin{equation}
\begin{gathered}
I_{ion}(V_m, w) = -kV_m(V_m-a)(V_m-1) -w
\\
g(V_m,w) = \epsilon(V_m -\gamma w)
\end{gathered}
\end{equation}

\vspace{5mm}

\subsection{Modello numerico semi-discretizzato}

	\vspace{5mm}
	 \begin{equation}
	 \begin{rcases}
	 V_{ij} &= \int_{\Omega}\nabla\varphi_j \cdot \nabla \varphi_i 
	 \\ I_{i,j}^T &= \sum_{F \in F_h^I} \int_{F} \{\{\nabla\varphi_j\}\} \cdot [[\varphi_i]] 
	\\ I_{i,j} &= \sum_{F \in F_h^I} \int_{F} [[\varphi_j]] \cdot \{\{\nabla \varphi_i\}\}
	 \\S_{i,j} &= \sum_{F \in F_h^I} \int_{F} \gamma [[\varphi_j]] \cdot [[\varphi_i]]]
	 \end{rcases}
	 \quad A = \Sigma(V -I^T - \delta I + S)
	\end{equation}
	\begin{equation}
	M_{ij} = \sum_{K \in \tau_h}\int_K
	\varphi_j\varphi_i
	\end{equation}
	\begin{equation}
	C(u_h)_{ij} = - \sum_{K \in \tau_h} \int_K \chi_m k(u_h-1)(u_h-a)\varphi_j\varphi_i
	\end{equation}
	\begin{equation}
	F_i = \int_{\Omega} f\varphi_i - \sum_{F in F_h^B} \int_F b\varphi_i
	\end{equation}
	
	\vspace{3mm}
\paragraph{Problema semi-discretizzato} 

\begin{equation*}
\begin{gathered}
\{\varphi_j\}_{j=1}^{N_h} \text{ base di } V_h^p = \{v_h \in L^2 : v_h|_K \in \mathbb{P}^{p_k}(K) \quad p_k \leq p \quad \forall K \in \tau_h \}
\\
u_h(t) = \sum_{j=1}^{N_h} u_j(t)\varphi_j, \quad w_h(t) = \sum_{j=1}^{N_h}w_j(t)\varphi_j
\end{gathered}
\end{equation*}
\vspace{3mm}
\begin{equation}
\Rightarrow \quad  \boxed{\chi_mC_m M \dot{u} +  A u + C(u_h) u -\chi_mMw=F}
\end{equation}
\vspace{5mm}
\subsection{Modello numerico totalmente discretizzato}
\vspace{5mm}
\paragraph{Forma implicita $(\theta \in [0,1])$}
\begin{enumerate}
	\item
	\begin{equation}
	\begin{gathered}
	\chi_mC_m M \frac{u^{k+1}-u^k}{\Delta t} +  A \left(\theta u^{k+1} + (1-\theta)u^k \right) + C(u^k)\left(\theta u^{k+1} + (1-\theta)u^k \right) +
	\\ -\chi_mMw^{k+1}=\theta F^{k+1} + (1-\theta)F^k
	\end{gathered}
	\end{equation}
	\item
	\begin{equation}
	\frac{w^{k+1}- w^{k}}{\Delta t} = \epsilon(u^k-\gamma w^{k+1})
	\end{equation}
\end{enumerate}
\vspace{3mm}
\paragraph{Forma esplicita $(\theta \in [0,1])$}
\begin{enumerate}
	\item
	\begin{equation}
	\begin{gathered}
	\left[\chi_mC_mM+\theta \Delta t A + \theta \Delta t C(u^k) \right] \bm{u^{k+1}} = \theta\Delta t F^{k+1} + (1-\theta)\Delta tF^k +
	\\
	\left[\chi_mC_mM- (1-\theta)\Delta t A - (1-\theta)\Delta t C(u^k)\right] u^k +\chi_m\Delta t Mw^{k+1} 
	\end{gathered}
	\end{equation}
	\item
	\begin{equation}
	\left[1+ \epsilon \gamma \Delta t\right] \bm{w^{k+1}} = w^k + (\epsilon \Delta t) u^k
	\end{equation}
\end{enumerate}







\newpage

\section{Bidominio}
\vspace{5mm}
\subsection{Modelli analitici}
\vspace{5mm}
\paragraph{Modello del bidominio}
\begin{equation}
\begin{cases}
\chi_m C_m\pdv{V_m}{t} - \nabla \cdot (\Sigma_i \nabla \phi_i) + \chi_m I_{ion}(V_m,w) = I_i^{ext}    & \text{in } \Omega_{mus} \cross (0,T]
\\
-\chi_m C_m\pdv{V_m}{t} - \nabla \cdot (\Sigma_e \nabla \phi_e) - \chi_m I_{ion}(V_m,w) = -I_e^{ext}    & \text{in } \Omega_{mus} \cross (0,T]
\\
\pdv{w}{t} = g(V\ped{m}, w)  & \text{in } \Omega_{mus} \cross (0,T]
\\
\Sigma_i\nabla \phi_i \cdot n = b_i   & \text{on } \partial \Omega_{mus} \cross (0,T]
\\
\Sigma_e\nabla \phi_e \cdot n = b_e   & \text{on } \partial \Omega_{mus} \cross (0,T]
\end{cases}
\end{equation}
\vspace{4mm}\newline
dove le incognite sono:
\begin{itemize}
	\item $\phi_i, \phi_e$  $\quad (V_m = \phi_i-\phi_e$)
	\item $w$  ("\emph{gating variable}")
\end{itemize}
e sono date le costanti : $\chi_m, C_m, \Sigma_i, \Sigma_e$

\paragraph{Modello di FitzHugh-Nagumo}
\begin{equation}
\begin{gathered}
I_{ion}(V_m, w) = -kV_m(V_m-a)(V_m-1) -w
\\
g(V_m,w) = \epsilon(V_m -\gamma w)
\end{gathered}
\end{equation}

\vspace{5mm}


\end{document}